\documentclass[oneside]{book}
%\documentclass[openany]{book}
%\documentclass{book}

\usepackage{graphicx}

\usepackage{amsmath}
\usepackage{mathtools}
\usepackage{amssymb}
\usepackage{comment}

\usepackage{amsthm}
\usepackage[T1]{fontenc}
\usepackage{libertine}

\usepackage[english]{babel}
\newtheorem{theorem}{Theorem}[section]
\newtheorem{corollary}{Corollary}[theorem]
\newtheorem{lemma}[theorem]{Lemma}
\newtheorem{mydef}{Definition}
\newtheorem{exmp}{Example}
\newtheorem{prop}{Proposition}

\date{\today}
\usepackage{lmodern}



\usepackage{hyperref}
\hypersetup{
    colorlinks=true, %true if colored links
    linktoc=all,     %all if both sections and subsections linked
    linkcolor=blue,  %color links
    }
    \usepackage[utf8]{inputenc}
\usepackage{csquotes}
\begin{document}

\begin{titlepage}
    \begin{center}
        \vspace*{1cm}
            
        \Huge
        \textbf{Isoperimetric Inequality}
            
        \vspace{0.5cm}
        \LARGE
    
            
        \vspace{1.5cm}
            
        \textbf{Prashant Kumar}
            
        \vfill
            
        \textit{A dissertation submitted for the partial fulfilment of BS-MS dual degree in Science}          
        \vspace{0.9cm}
            
        \includegraphics[width=0.4\textwidth]{iiser logo.png}
            
        \Large
        Department of Mathematical Sciences\\
        Indian Institue of Science Education and Research Mohali\\
        India\\
        June 2020
            
    \end{center}
\end{titlepage}




\urlstyle{same}
\tableofcontents
\thispagestyle{empty}
 \addcontentsline{toc}{chapter}{Certificate of Examination}
\chapter*{Certificate of Examination}
This is to certify that the dissertation titled \enquote{Isoperimetric Inequality} submitted by Prashant Kumar (Reg. number $-$ MS15114) for the partial fulfillment of BS-MS dual degree program of the Institute, has been examined by the thesis committee duly appointed by the Institute. The committee finds the work done by the candidate satisfactory and recommends that the report be accepted.\\\\
\textbf{Dr. Soma Maity} (Supervisor)\\
\\
 \textbf{Dr. Lingaraj Sahu}  \hfill           \textbf{ Dr. Pranab Sardar}\\
\\
\\
\rightline{\textbf{Date:} April 2020}
\thispagestyle{empty}
 \addcontentsline{toc}{chapter}{Declaration of Authorship}
 \chapter*{Certificate of Examination} 
\pagenumbering{roman}
This is to certify that the dissertation titled \enquote{Isoperimetric Inequality} submitted by Prashant Kumar (Reg. number $-$ MS15114) for the partial fulfillment of BS-MS dual degree program of the Institute, has been examined by the thesis committee duly appointed by the Institute. The committee finds the work done by the candidate satisfactory and recommends that the report be accepted.\\\\
\textbf{Dr. Soma Maity} (Supervisor)\\
\\
 \textbf{Dr. Lingaraj Sahu}  \hfill           \textbf{ Dr. Pranab Sardar}\\
\\
\\
\rightline{\textbf{Date:} April 2020}
\thispagestyle{empty}
\chapter*{Declaration of Authorship}
The work presented in this dissertation has been carried out by me under the guidance of \textbf{Dr. Soma Maity} at the Indian Institute of Science Education and Research, Mohali.
This work has not been submitted in part or in full for a degree, a diploma, or a fellow- ship to any other university or institute. Whenever contributions of others are involved, every effort is made to indicate this clearly, with due acknowledgement of collaborative research and discussions. This thesis is a bonafide record of original work done by me and all sources listed within have been detailed in the bibliography.\\
\\  In \ref{chap :c2}
\textbf{Prashant Kumar} (Candidate)\\
Date: April 2020\\
\\
\\
\\
In my capacity as the supervisor of the candidate$'$s project work, I certify that the above statements by the candidate are true to the best of my knowledge.\\
\\
\textbf{Dr. Soma Maity}(Supervisor)\\
Date: April 2020
\thispagestyle{empty}
\chapter*{Acknowledgements}
Thank you.
\thispagestyle{empty}

\addcontentsline{toc}{chapter}{Abstract}
\chapter*{Abstract}









































\chapter{Introduction}
\label{chap:c1}
\pagenumbering{arabic}
 Word "Isoperimetric" means having same perimeter. Isoperimetric Inequality  is a geometric inequality which relates perimeter of a set and its volume. classical Isopermetric inequality in the plane is that Among all closed curves in the plane of fixed perimeter, which curve if it exists, maximizes the area of its enclosed region? or in other way, Among all closed curves in the plane enclosing a fixed area, which curve minimizes the perimeter?\\
Isoperimetric Inequalities are defined for various space like for Ecludian space and Reimannian manifolds. In ecludian space sharp Isoperimetric Inequality are found with equality while in Reimannian manifolds Isoperimetric Inequality are not exact but are close enough to get its global geometric information. Here We will be discussing only  Isoperimetric Inequality in Ecludian spaces e.g. in  $\mathbb{R}^{n}$ and then specifically in convex subsets of $\mathbb{R}^{n}$.
\\


















\chapter{Preliminaries }
\label{chap:c2}
hyper plane , half space etc 
\begin{mydef}
$K$ and $L$ are said to be Homothetic, iff for some $x  \in \mathbb{R}^n,$ $K = \alpha L + x$ or $L=\alpha K+x $.\\
Note that if one of $K,L $ is a point then $K $ and $L$ are homothetic.
\end{mydef}
\label{def:2.1}
 \chapter{Isoperimetric Inequality in $\mathbb{R}^{n}$} 
 \label{chap:c3}
 Isoperimetric problem is to finding the domain which contains the greatest area on considering  all bounded domains with fixed given perimeter.
 

\section{\textbf{Basic Definitions}}\label{s:2}
\subsection{Curvature}
\label{ss:1}
    
For any $C^{2}$ path $\omega:(\alpha, \beta) \rightarrow \mathbb{R}^{2}$, its velocity vector field is given by its derivative $w'$ and its aceeleration vector field by its second derivative $w''$ assuming w is an immmersion($w'$ never vanishes ) in the plane.  Small infinitesimal element of Arc length is given by  $d s=\left|\omega^{\prime}(t)\right| d t$.
\\Unit tangent vector field along $\omega$ is $\mathbf{T}(t)$ defined as \\

\begin{equation}
\label{eq1}  
\textbf{{T}}(t)=\frac{\omega^{\prime}(t)}{\left|\omega^{\prime}(t)\right|}
\end{equation}  \\
and unit normal vector field $\mathbf{N}$ along $\omega$ is defined as 
\begin{equation}
\label{eq2}  
     \mathbf{N}=\mathbf{\tau} \mathbf{T}
\end{equation} \\
   where $ \tau: \mathbf{R}^{2} \rightarrow \mathbf{R}^{2}$ is the rotation of $\mathbb{R}^{2}$ by $\pi / 2$ radians, and

  Then its curvature $\kappa$ is defined as
  \begin{equation}
  \label{eq3}  
    \frac{d \mathbf{T}}{d s}=\kappa \mathbf{N}
\end{equation}  and similarly it turns out that \begin{equation}
\label{eq4}  
    \frac{d \mathbf{N}}{d s}=-\kappa \mathbf{T}
\end{equation}

\subsection{Relative Compact Domain} \label{ss:2}
a relatively compact subspace of a topological space X is a subset whose closure is compact.
every subset of a compact space is relatively compact since
    \







\section{\textbf{Isoperimetric Inequality in the Plane($\mathbb{R}^{2}$)}}
\label{s:3}
 In $\mathbb{R}$, Discrete measure of boundary of any bounded open subset of the line is greater than or equal to 2, and equal to 2 when given bounded open subset is just an open interval.\\
  In $\mathbb{R}^{2}$ Isoperimetric problem is to finding the domain which contains the greatest area on considering  all bounded domains with fixed given perimeter. For $\mathbb{R}^{2}$ it is disk . 

    If area of domain is A and length of its boundary is L then using values of perimeter and area of disk,
    as an analytical inequality it can be written as \\\\
    \begin{equation}
    \label{eq5}  
    L^{2} \geq 4 \pi A
        \end{equation}
      \\
        In $\mathbb{R}^{n}$ 
        , $n \geq 2$  it can be generalised as
        \\\\
       \begin{equation}
       \label{eq6}  
             \frac{A(\partial \Omega)}{V(\Omega)^{1-1 / n}} \geq \frac{A\left(S^{n-1}\right)}{V\left(\mathbb{B}^{n}\right)^{1-1 / n}}
          \end{equation} \\\\
        where $\Omega$ is any bounded domain in $\mathbb{R}^{n}$ and $\partial \Omega$ its boundary, $V$ denotes $(n-1)$
measure and $A$ denotes $(n-1)$ -measure, $B^{n}$ is the unit disk in $R^{n},$ and $S^{n-1}$,
the unit sphere in $R^{n}$.
Let $\omega_{n}$ denote the $(n-1)$ -dimensional volume of $B^{n}$ and  $c_{n-1}$ the $(n-1)$ -dimensional surface area of $S^{n-1}$ \\\\



We have standard result of $\omega_{n}$ and $c_{n-1}$ for $\mathbb{R}^{n}$ as  \\\\
    \begin{equation}
    \label{eq7}  
\mathbf{c}_{\mathbf{n}-1} = \frac{2 \pi^{n / 2}}{\Gamma(n / 2)}
    \end{equation}
    \\\\
        \begin{equation}
        \label{eq8}  
 \omega_{\mathbf{n} } = \frac{\mathbf{c}_{\mathbf{n}-1}}{n}  
    \end{equation} \\
that gives us \\
    \begin{equation}
    \label{eq9}  
 \frac{A(\partial \Omega)}{V(\Omega)^{1-1 / n}} \geq n \omega_{\mathbf{n}}^{1 / n}
    \end{equation}






\begin{theorem}
{(Uniqueness for Smooth Boundaries)}
\label{t:1} \\
 Given
the area $A,$ let $D$ vary over relatively compact domains in the plane of area $A$
with $C^{1}$ boundary, and suppose the domain $\Omega, \partial \Omega \in C^{2}$, realizes the minimal
boundary length among all such domains $D .$ We claim that $\Omega$ is a disk.
\end{theorem}


\begin{proof}

 since $\Omega$ is relatively compact in $R^{2},$ there exists a simply connected
domain $\Omega_{0}$ such that
$$
\Omega=\Omega_{0} \backslash\{\text{finite disjoint union of closed topplogical disks} $$
 but $\Omega_{0}=\Omega $ otherwise on adding the
topological disks to $\Omega$, will increase its area  and decrease the lenghth of boundary and so $\Omega$ will no longer be minimizer. So $\Omega_{0}=\Omega,$ and is bounded by an imbedded circle.\par
Let $\Gamma: \mathbf{S}^{1} \rightarrow \mathbb{R}^{2} \in C^{2}$ be the imbedding of the boundary of $\Omega .$ 
consider a 1 -parameter family $\Gamma_{\epsilon}: \mathbf{S}^{2} \rightarrow \mathbb{R}^{2}$ of imbeddings
\par
$$
v:\left(-\epsilon_{0}, \epsilon_{0}\right) \times \mathbf{S}^{1} \rightarrow \mathbb{R}^{2}
$$ \\
such that the  $v(\epsilon, t)$ is given by \\
    \begin{equation}
    \label{eq10}  
v(\epsilon, t)=\Gamma_{\varepsilon}(t)=\Gamma(t)+\Psi(\epsilon, t) \nu(t), \quad \Psi(0, t)=0
    \end{equation} \\
is $C^{1} .$ We have \\\\
  \begin{equation}
  \label{eq11}  
\frac{\partial v}{\partial \epsilon}=\frac{\partial \Psi}{\partial \epsilon} \nu
    \end{equation}
 \\ 
 and 
     \begin{equation}
     \label{eq12}  
    \\  \frac{\partial v}{\partial t}=\Gamma^{\prime}+\left\{\frac{\partial \Psi}{\partial t} \nu+\Psi \nu^{\prime}\right\}=\{1+\kappa \Psi\} \Gamma^{\prime}+\frac{\partial \Psi}{\partial t} \nu
                   \end{equation}    
which implies  

                    $$\left|\frac{\partial v}{\partial t}\right|=\left\{(1+\kappa \Psi)^{2}+\frac{1}{\left|\Gamma^{\prime}\right|^{2}}\left(\frac{\partial \Psi}{\partial t}\right)^{2}\right\}^{1 / 2}\left|\Gamma^{\prime}\right|$$
Let   \\
                     
                      $$\phi(t):=\left.\frac{\partial \Psi}{\partial \epsilon}\right|_{\epsilon=0}$$ \\\\
                      
expanding $\Psi(\epsilon, t)$ aroud $\epsilon$ using Tailor series expansion we get \\

$$
\Psi(\epsilon, t)=\epsilon \phi(t)+o(\epsilon), \quad \frac{\partial \Psi}{\partial \epsilon}=\phi(t)+o(1), \quad \frac{\partial \Psi}{\partial t}=O(\epsilon)
$$ \\
On simplifying and ignoring $O^2(\epsilon)$ and its higher order term we will get \\\\
$$
\left|\frac{\partial v}{\partial t}\right|=\left|\Gamma^{\prime}\right|\{1+\epsilon x \phi+o(\epsilon)\}
$$ 
\\\\
 the Area element $d A$  is given by
 \\\\
     \begin{equation}
     \label{eq13}  
d A=\left|\frac{\partial v}{\partial \epsilon} \times \frac{\partial v}{\partial t}\right| d \epsilon d t=\phi\left|\Gamma^{\prime}\right|\{1+o(1)\} d \epsilon d t=\{\phi+o(1)\} d \epsilon d s
    \end{equation} \\\\
We have $A\left(\Omega_{\epsilon}\right)=A(\Omega)$ for all $\epsilon$, so for small $\epsilon$ \\

 $$
A\left(\Omega_{\epsilon}\right)-A(\Omega)=\int_{0}^{\epsilon} d \sigma \int_{\Gamma}[\phi+o(1)] d s
$$
\\\\
So we have \\
$$  
\int_{\Gamma} \phi d s=0
$$ \\\\
 Let $L(\epsilon)$ denote the length of $\Gamma_{e} .$  we have
$L^{\prime}(0)=0$ because since $\Gamma$ has the shortest length, Therefore, because
$L(\epsilon)=\int_{\mathbf{S^1}}\left|\frac{\partial v}{\partial t}\right| d t=\int_{\mathbf{S}^{1}}\left|\Gamma^{\prime}\right|\{1+\epsilon \kappa \phi+o(\epsilon)\} d t=\int_{\Gamma}\{1+\epsilon \kappa \phi+o(\epsilon)\} d s$
 so if we have
 $\quad \int_{\Gamma} \phi d s=0$ then 
$$
L^{\prime}(0)=\int_{\Gamma} \kappa \phi d s=0, 
$$

\end{proof} 

\begin{theorem}
\label{t:2}

 \textbf{(Isoperimetric Inequality in $\mathbb{R}^{2}$ )} Let $\Omega$ be a relatively compact domain in $\mathbb{R}^{2}$, with boundary $\partial \Omega \in C^{1}$ Then
      \begin{equation}
      \label{eq14}  
   L^{2}(\partial \Omega) \geq 4 \pi A(\Omega) 
      \end{equation}
 with equality when $\Omega$ is a disk.\\\\

\begin{proof}

 Let $x=x^{1} e_{1}+x^{2} e_{2}$ be the vector field on $R^{2}$ with base point $\textbf{x}=\left(x^{1}, x^{2}\right) .$
We have 2-dimensional divergence theorem for any vector
field $\textbf{x} \mapsto \xi(\textbf{x}) \in \mathbb{R}^{2}$ with support containing cl $\Omega$.
\\\\
    \begin{equation}
    \label{eq15}  
\int_{\Omega} \operatorname{div} \boldsymbol{\xi} d A=\int_{a \Omega} \boldsymbol{\xi} \cdot \nu d s 
  \end{equation} \\
  
where $\nu$ denote outward unit normal vector along $\partial\Omega$. \\
    
For  $ x=x^{1} e_{1}+x^{2} e_{2}$ we have 
$\operatorname{div} \textbf{x}=2$ on all $\Omega$. 
\\\\\\
So we have
\\
$$ 2 A(\Omega)=\int_{\Omega} \operatorname{div} x d A=\int_{\partial \Omega} x \cdot \nu d s$$ 
Using vector-Schwartz inequality we have 

$$ \int_{\partial \Omega} \mathbf{x} \cdot \nu d s \leq \int_{\partial \Omega} | \mathbf{x}| d s  $$ 
\\\\
and now using integral cauchy-schwarz inequality. \\\\
$$\int_{\partial \Omega}|\mathbf{x}| d s \leq\left\{\int_{\partial \Omega}|\mathbf{x}|^{2} d s\right\}^{1 / 2}\left\{\int_{\partial \Omega} 1^{2} d s\right\}^{1 / 2} =L^{1 / 2}(\partial \Omega)\left\{\int_{\partial \Omega}|\mathbf{x}|^{2} d s\right\}^{1 / 2}$$ \\\\
We have \\\\
$|\mathbf{x}|^{2} = \left(x^{1}\right)^{2}+\left(x^{2}\right)^{2}, \quad\left|\frac{d \mathbf{x}}{d s}\right|^{2}=\left(\frac{d x^{1}}{d s}\right)^{2}+\left(\frac{d x^{2}}{d s}\right)^{2}$\\\\
applying Wirtinger's inequality to each coordinate function $x^{1}(s)$ and $x^{2}(s)$ implies \\\\
$$ 2 A(\Omega) \leq L^{1 / 2}(\partial \Omega)\left\{\int_{\partial \Omega}|\mathbf{x}|^{2} d s\right\}^{1 / 2} \leq L^{1 / 2}(\partial \Omega)\left\{\frac{L^{2}(\partial \Omega)}{4 \pi^{2}} \int_{\partial \Omega}\left|\mathbf{x}^{\prime}\right|^{2} d s\right\}^{1 / 2}$$
\\
$$=\frac{L^{2}(\partial \Omega)}{2 \pi}$$ \\
So we have $$L^{2}(\partial \Omega) \geq 4 \pi A(\Omega)$$\\
Equality follows easily as for the disk $L^{2}(\partial \Omega) = 4 \pi A(\Omega)$ \\

\end{proof}
\end{theorem}



\section{\textbf{Definitions}}
\label{s:4}
\subsection{Convex Sets}
    \label{ss:3}
 A set $A$ in $R^{n}$ is \textbf{Convex} if $x, y \in A$ implies that $\lambda x+(1-\lambda) y \in A$
for all $\lambda \in(0,1),$ that is, for any $x$ and $y$ in $A$ the closed line segment $[x, y]$ in
$R^{n}$ joining them is contained in $A .$\\~\\

 A Convex linear combination of elements $x_{1}, \ldots, x_{k} \in \mathbb{R}^{n}$ is the
linear combination
$
\sum_{j=1}^{k} \lambda_{j} x_{j}
$
where the coefficients satisfy
$$
\sum_{j=1}^{k} \lambda_{j}=1, \quad \lambda_{j} \geq 0 \vee j
$$


If $A$ is convex then any convex linear combination of elements
of $A$ is a point in $A .$\\

 
 
 Given $A \subset R^{n}$, \textbf{Convex Hull} of $A,$ conv $A,$ is the smallest
convex set containing $A .$


 \subsection{Hyperplane} 
 \label{ss:4}
 In general, the word “hyperplane” refers to an $\{n-1\}$ dimensional flat in $R^{n}$ \\
 Hyperplane is a subspace whose dimension is one less than that of its ambient space  
 \\ it is preimage of a function from $\mathbb{R}^{n}$ to $\mathbb{R}$ i.e.
 \begin{equation}
 \label{eq16}  
   H=\{v \in V: \alpha \cdot v=0\} 
  \end{equation}  
    or simply  $H = \{f = \alpha \}$. 
   it can be written	as  a linear equation of the form
$$a_1x_1 + a_2x_2 + ... + a_nx_n = b,$$ where $a_1,a_2,..a_n$ are constants and $(a_1,a_2,a_3,...,a_n)$ is normal vector to the hyperplane.
\\\\
 Two half-spaces  determined by hyperplane H are 
$H^{-}=\{v \in V: \alpha \cdot v \leq 0\}$ and  $ H^{+}=\{v \in V: \alpha \cdot v \geq 0\} $ 
\\
Let $A, B$ be subsets of $R^{n},$ and $H=\{f=\alpha\}$ be a hyperplane. We say $A$ and
$B$ are separated by $H$ if $A$ and $B$ lie in different closed half-spaces deter
mined by $H .$ If neither $A$ nor $B$ intersects $H,$ we say $H$ stricty separates $A$
and $B .$\\

\begin{mydef}
Let $A$ be a subset of $R^{n}$ which is closed and convex. We say hyperplane \\ $H=\{f=\alpha\}$ is \textbf{supporting hyperplane} of $A$ at  $x \in H$  if $A \cap E \neq \emptyset$ and $A$ is contained in any of the two closed half-spaces $\{f \leq \alpha\}, \{f \geq \alpha\}$.
\\
    Supporting half-space of $A$ is half-space which contains $A$ and is bounded by supporting hyperplane of $A$  and the set $A \cap H$ is called \textbf{Support Set} and any of its point $x \in A \cap H$ is called supporting point.
\end{mydef}

\subsection{Hypersurface}
\label{ss:5}



A Hypersurface is a manifold of dimension $(n-1)$, which is embedded in an ambient space of dimension n,generally a Euclidean space.
\\
    If M and N are differentiable manifolds such that $dim(M)-dim(N) = 1$ and if an immersion $f:N\rightarrow M$ is defined then $f(N)$ is a hypersurface in M, here f is a differentiable mapping whose differential \textbf{df} at any point  $x\in M$ is an injective mapping of the 
     tangent space $N_{x}$ of N to  tangent space $M_{f(x)}$ of M at $f(x)$.
    

    
    
       Assume  Hypersurface $\Gamma$ is given locally by the $C^{1}$ mapping $\mathbf{x}: G \rightarrow \mathbf{R}^{n}$ of everywhere
maximal rank, where $G$ is an open subset of $\mathbb{R}^{n-1} .$ So $\mathbf{x}=\mathbf{x}(u)$; and the vectors \\
$$
\partial \mathbf{x} / \partial u^{1}, \ldots, \partial \mathbf{x} / \partial u^{n-1} 
$$
are linear indepedent because  mapping $\mathbf{x}$ is of maximal rank and these vectors span the tangent space to $\Gamma$ at every $x(u)$\\







   


\subsection{Riemannian metric and First Fundamental Form}
\label{ss:6}
{Riemannian metric} of $\Gamma$ is given
locally by the positive definite matrix $G(u),$ where 
\\\\
  \begin{equation}
  \label{eq17}  
 G=\left(g_{j k}\right), \quad g_{j k}=\frac{\partial \mathbf{x}}{\partial u^{j}} \cdot \frac{\partial \mathbf{x}}{\partial u^{k}}, \quad j, k=1, \ldots, n-1     \end{equation}
\\
it is also called \textbf{First Fundamental Form.} \\\\
   We use these Notation \\
   \[ G^{-1}=\left(g^{j k}\right),   g=\operatorname{det}{G} \] \\
    And the associated surface area on  
   $\Gamma$ is given locally by \\


   \begin{equation}
   \label{eq18}  
    d A=\sqrt{g} d u^{1} \cdots d u^{n-1} 
\end{equation}

   
  







\subsection{Reimannian Divergence}
\label{ss:7}
Let $\Gamma$ is a $C^{2}$ hypersurface in $\mathbf{R}^{n}$,
For any tangent vector field $\zeta$ along $\Gamma$, we can write 
$$
\zeta=\sum_{j=1}^{n-1} \zeta^{j} \frac{\partial \mathbf{x}}{\partial u^{j}}
$$
and for this its \textbf{Reimannian Divergence} is given by \\\\
\begin{equation}
\label{eq19}  
     \operatorname{div}_{\mathrm{r}} \boldsymbol{\zeta}=\frac{1}{\sqrt{g}} \sum_{j=1}^{n-1} \frac{\partial\left(\zeta^{j} \sqrt{\boldsymbol{g}}\right)}{\partial \boldsymbol{u}^{j}}  
\end{equation}
     
    
    
    



    
    
    
      For given any $(n-1)$ - dimensional domain $\Lambda \subset \Gamma$,  with $C^{1}$ boundary $\partial \Lambda,$ which is $(n-2)$-dimension and unit normal exterior vector field $\nu$ along $\partial \Lambda$, \textbf{Riemannian divergence theorem} is 
      
        \begin{equation}
        \label{eq20}  
      \operatorname{div}_{\mathrm{r}} \zeta=\frac{1}{\sqrt{g}} \sum_{j=1}^{n-1} \frac{\partial\left(\zeta^{j} \sqrt{\boldsymbol{g}}\right)}{\partial \boldsymbol{u}^{j}}
        \end{equation}
      
      
      \subsection{Second Fundamental Form}
      \label{ss:8} Second Fundamental Form of $\Gamma$ in $\mathbb{R}^{n}$ is given locally by 
        \begin{equation}
        \label{eq21}  
      \boldsymbol{B}=\left(\boldsymbol{b}_{j k}\right), \quad b_{j k}=\frac{\partial^{2} \mathbf{x}}{\partial \boldsymbol{u}^{j} \partial \boldsymbol{u}^{k}} \cdot \mathbf{n}, \quad j, k=1, \ldots, \boldsymbol{n}-1 
        \end{equation}
      
      $$b_{j k}=\frac{\partial^{2} \mathbf{x}}{\partial u^{j} \partial u^{k}} \cdot \mathbf{n}=\frac{\partial}{\partial u^{j}}\left(\mathbf{n} \cdot \frac{\partial \mathbf{x}}{\partial u^{k}}\right)-\frac{\partial \mathbf{n}}{\partial u^{j}} \cdot \frac{\partial \mathbf{x}}{\partial u^{k}}=-\frac{\partial \mathbf{n}}{\partial u^{j}} \cdot \frac{\partial \mathbf{x}}{\partial u^{k}}$$ \\\\
      
      
       as $\boldsymbol{n}$ and $\frac{\partial u_{j}}{\partial x} $ are perpendicular
      So it can also be written as 
        \begin{equation}
        \label{eq22}  
       b_{j k} = -\frac{\partial \mathbf{n}}{\partial u^{j}} \cdot \frac{\partial \mathbf{x}}{\partial u^{k}}         \end{equation} \\
       where $\mathbf{n}$ is normal unit vector field along $\Gamma$ \\

      
      The \textbf{Mean Curvature} $H$ of $\Gamma$ in $\mathbb{R}^{n}$ is the trace of $\mathcal{L} = G^{-1} B$, which is $\operatorname{tr} G^{-1} B$, the trace of $\mathcal{B}$ relative
to $G,$ given by
$
\boldsymbol{H}=\operatorname{tr} G^{-1} B
$ and \textbf{Gauss-Kronekar Curvature} is given by $\boldsymbol{K}=\operatorname{det} {G}^{-1} {B}$





\section{Isoperimetric Inequality in domains with $C^{2}$ \\boundary} 
\label{s:5}
In this section we will be studying isoperimetric Inequality in domains with $C^{2}$ boundary.Using classical arguments we show that if a domain gives a solution to $C^{2}$  Isoperiometric problem then it must be a diak and then to show if domain is extremal of Isoperimetric functional then it must be a disk.



















\begin{theorem}
\label{t:3}
 Let $\Omega$ be a bounded domain in $\mathbb{R}^{n}$, with $C^{2}$ boundary $\Gamma$. Given
any $C^{2}$ time-dependent vector field $X: \mathbf{R}^{n} \times \mathbb{R} \rightarrow \mathbb{R}^{n}$ on $\mathbf{R}^{n},$ let $\Phi_{t}: \mathbf{R}^{n} \rightarrow \mathbb{R}^{n}$
denote the l-parameter flow determined by $X\\ \Phi_{t}$ and $X$ are related by $$\frac{d}{d t} \Phi_{t}(x)=X(x, t), \quad \Phi_{0}=\mathrm{id.}$$
 and $$\xi(x)=X(x, 0), \quad \eta=\xi | \Gamma$$
 Then 
 \begin{equation}
  \label{eq23}
      {\textbf{(i)}}\ \left.\frac{d}{d t} \mathbf{V}\left(\Phi_{t}(\Omega)\right)\right|_{t=0}=\iint_{\Omega} \operatorname{div} \xi d \mathbf{v}_{n}=\int_{\Gamma} \boldsymbol{\eta} \cdot \mathbf{n} d A
  \end{equation}    
    \\
    
    
 \begin{equation}
  \label{eq24}
 {\textbf{(ii)}}\  \left.\frac{d}{d t} A\left(\Phi_{t}(\Gamma)\right)\right|_{t=0}=\int_{\Gamma}\left\{\operatorname{div}_{\Gamma} \boldsymbol{\eta}^{T}-H \boldsymbol{\eta} \cdot \mathbf{n}\right\} d \boldsymbol{A}=-\int_{\Gamma} H \boldsymbol{\eta} \cdot \mathbf{n} d \boldsymbol{A}
 \end{equation}
 


 Where n is chosen to exterior normal vector field and ${\eta}^{T}$ is tangential part of  $\eta$.\\\\
 

\begin{proof}: \textbf{(i)}  \\
If $J_{\phi}$, denotes the Jacobian marrix of $\Phi_{t}$.Then we have \\

 
\begin{equation}
 \label{eq25}
    V\left(\Phi_{t}(\Omega)\right)=\iint_{\Omega} \operatorname{det} J_{\phi}(x) d v_{n}(x)
     \end{equation} 
    
so we have 

 \begin{equation} 
  \label{eq25}
\frac{d}{d t} V\left(\Phi_{t}(\Omega)\right)=\iint_{\Omega}\left(\frac{d}{d t} \operatorname{det} J_{\phi}(x)\right) d v_{n}(x)
\end{equation}
\\\\
For any differentiable
matrix function $t \mapsto \mathcal{A}(t),$ where $\mathcal{A}(t)$ is nonsingular we have \\\\
\begin{equation}
 \label{eq26}
\frac{d}{d t} \operatorname{det} \mathcal{A}=\operatorname{det} \mathcal{A} \cdot \operatorname{tr}\left(\mathcal{A}^{-1} \frac{d \mathcal{A}}{d t}\right)
\end{equation}
Therefore \\\\
$$\frac{d}{d t} V\left(\Phi_{t}(\Omega)\right)=\iint_{\Omega}\left(\operatorname{det} J_{\phi_{t}}\right) \cdot \operatorname{tr}\left(J_{\phi_{t}^{-1}}\frac{d}{d t} J_{\phi_{t}}\right) d v_{n}(x)$$
\\
Now  $J_{\phi_{t}}$ is Jacobian matrix so \\\\
\begin{equation}
 \label{eq27}
\left(J_{\phi_{t}}\right)_{A}^{B}=\frac{\partial \Phi_{t}^{B}}{\partial x^{A}}, \quad A, B=1, \ldots, n  
\end{equation} 
\\\\
and at $t = 0$ we have $\phi_{0}(x) = x$ or $\phi_{0}$ is identity so 
\\

$$\left.\frac{\partial \Phi_{t}^{B}}{\partial x^{A}}\right|_{t=0}=\delta_{A}^{B}$$ where $\delta$ is kronekar delta.\\\\ 
Furthermore \\
    $$\frac{d}{d t}\left(J_{\phi}\right)_{A}^{B}=\frac{\partial}{\partial t} \frac{\partial \Phi_{t} B}{\partial x^{A}}=\frac{\partial}{\partial x^{A}} \frac{\partial \Phi_{t}^{B}}{\partial t}$$ \\


   and at $t=0$, which implies 
   \begin{center}
       $$\left.\frac{d}{d t}\left(J_{\phi}\right)_{A}^{B}\right|_{t=0}=\frac{\partial \xi^{B}}{\partial x^{A}}$$ \\
   \end{center}

 Since \\
  $$\left.\frac{\partial \Phi_{t}^{B}}{\partial x^{A}}\right|_{t=0}=\delta_{A}^{B} $$
  \\
   so  $J_{\phi_{t}} $ at $t = 0,$ is an Identity matrix and so its inverse, and  $\operatorname{det} J_{\phi_{0}} = 1$\\\\
 So \\
 \begin{equation}
  \label{eq28}
  \left.\frac{d}{d t} \mathbf{V}\left(\Phi_{t}(\Omega)\right)\right|_{t=0} = \iint_{\Omega}\operatorname{tr}\left(\frac{d}{d t} J_{\phi_{t}}\right) d v_{n}(x) = \iint_{\Omega} \operatorname{div} \xi d \mathbf{v}_{n} \\\\
  \end{equation} \\\\
  Now using divergence theorem we get \\\\
 \begin{equation}
  \label{eq29}
  \left.\frac{d}{d t} V\left(\Phi_{t}(\Omega)\right)\right|_{t=0}=\iint_{\Omega} \operatorname{div} \xi d \mathbf{v}_{n}=\int_{\Gamma} \eta \cdot \mathbf{n} d A \\\\
 \end{equation}

 
 
 \end{proof}
 \begin{proof}
 
 
 \textbf{: (ii)} 
 \\
 Assume the surface $\Gamma$ is given locally by $\mathbf{x}=\mathbf{x}(u)$, and take \\
$$
\mathbf{y}(u, t)=\Phi_{t}(\mathbf{x}(u))
$$
Denote $\phi = \eta\cdot{n}$ and the Riemannian metric on $\Phi_{t}(\Gamma)$ for each fixed $t$ by 
\begin{equation}
 \label{eq30}
h_{j k}=\frac{\partial \mathbf{y}}{\partial u^{j}} \cdot \frac{\partial \mathbf{y}}{\partial u^{k}}, \quad j, k=1, \ldots, n-1
\end{equation} \\
On $\Gamma$ with reimannian metric \textbf{G} and  $g=det(\textbf{G})$, associated surface area is given locally by 

\begin{equation}
\label{eq31}
    d A=\sqrt{g} d u^{1} \cdots d u^{n-1}
\end{equation} 
So we have \\

$$
\frac{d}{d t} A\left(\Phi_{t}(\Gamma)\right)=\int_{\Gamma} \frac{\partial}{\partial t} \sqrt{\operatorname{det}\left(h_{j k}\right)} d u^{1} \cdots d u^{n-1}$$ \\\\
In calculation of  derivative of $\sqrt{\operatorname{det}\left(h_{j k}\right)}$, set $\mathcal{H}=\left(h_{j k}\right), \mathcal{H}^{-1}=\left(h^{j k}\right) $ \\\\ then \\
$$
\frac{\partial}{\partial t} \sqrt{\operatorname{det} \mathcal{H}}=\frac{1}{2}\{\sqrt{\operatorname{det} \mathcal{H}}\}^{-1} \operatorname{det} \mathcal{H} \cdot\left(\operatorname{tr} \mathcal{H}^{-1} \frac{\partial \mathcal{H}}{\partial t}\right)=\frac{1}{2} \sqrt{\operatorname{det} \mathcal{H}} \sum_{j, k}{h}^{j k} \frac{\partial h_{\boldsymbol{k} j}}{\partial t}$$  \\
  using 
 $$\frac{d}{d t} \operatorname{det} \mathcal{H}=\operatorname{det} \mathcal{H} \cdot \operatorname{tr}\left(\mathcal{H}^{-1} \frac{d \mathcal{H}}{d t}\right) $$.  
 \\\\
Now putting  $h_{k j}=\frac{\partial \mathbf{y}}{\partial u^{k}} \cdot \frac{\partial \mathbf{y}}{\partial u^{j}}$ 
 We get \\

 
$$\frac{\partial}{\partial t} \sqrt{\operatorname{det} \mathcal{H}}=\sqrt{\operatorname{det} \mathcal{H}} \sum_{j, \boldsymbol{k}} \boldsymbol{h}^{j k} \frac{\partial \mathbf{y}}{\partial \boldsymbol{u}^{k}} \cdot \frac{\partial}{\partial \boldsymbol{u}^{j}} \frac{\partial \mathbf{y}}{\partial t}$$  \\

  At t=0 ,  for  $\eta(u)=(\partial \mathbf{y} / \partial t)(u, 0)$, Along $\Gamma$, We can write \\
  
  \begin{equation}
  \label{eq32}  \eta=\sum_{\ell=1}^{n-1} \eta^{\ell} \frac{\partial \mathbf{x}}{\partial u^{\ell}}+\phi \mathbf{n} 
  \end{equation}
  \\
where 
$\phi = \eta\cdot{n}$, for t=0, $\mathbf{y}(u, 0)=\mathbf{x}(u)$ so $\mathcal{H}=G$ hence we have\\\\ 
$$\left.\frac{\partial}{\partial t} \sqrt{\operatorname{det} \mathcal{H}}\right|_{t=0}=\sqrt{\operatorname{det} G} \sum_{j, k} g^{j k} \frac{\partial \mathbf{x}}{\partial u^{k}} \cdot \frac{\partial \eta}{\partial u^{j}}$$ and \\\\
$$\frac{\partial \boldsymbol{\eta}}{\partial \boldsymbol{u}^{j}}=\sum_{\ell=1}^{n-1}\left\{\frac{\partial \eta^{\ell}}{\partial \boldsymbol{u}^{j}} \frac{\partial \mathbf{x}}{\partial \boldsymbol{u}^{\ell}}+\eta^{\ell} \frac{\partial^{2} \mathbf{x}}{\partial \boldsymbol{u}^{j} \partial \boldsymbol{u}^{\ell}}\right\}+\frac{\partial \boldsymbol{\phi}}{\partial \boldsymbol{u}^{j}} \mathbf{n}+\boldsymbol{\phi} \frac{\partial \mathbf{n}}{\partial \boldsymbol{u}^{j}}$$ \\\\

So since $\mathbf{n}$ is perpendicular to $\partial \mathbf{x} / \partial u^{k}$ for all $k=1, \ldots, n-1$
\\\\
$$\sum_{j, k} g^{j k} \frac{\partial \mathbf{x}}{\partial u^{k}} \cdot \frac{\partial \eta}{\partial u^{j}}=\sum_{j, k} g^{j k} \frac{\partial \mathbf{x}}{\partial u^{k}} \cdot\left(\sum_{\ell=1}^{n-1}\left\{\frac{\partial \eta^{\ell}}
{\partial u^{j}} \frac{\partial \mathbf{x}}{\partial u^{\ell}}+\eta^{\ell} \frac{\partial^{2} \mathbf{x}}{\partial u^{\ell} \partial u^{j}}\right\}+\phi \frac{\partial \mathbf{n}}{\partial u^{j}}\right)$$ \\\\
$${=\sum_{j, k, \ell} g^{j k} g_{k \ell} \frac{\partial \eta^{\ell}}{\partial u^{j}}+\sum_{j, k, \ell} g^{j k} \eta^{\ell}
\frac{\partial \mathbf{x}}{\partial u^{k}} \cdot \frac{\partial^{2} \mathbf{x}}{\partial {u}^{\ell} \partial {u}^{j}}
+\sum_{j, k} \phi g^{j k} \frac{\partial \mathbf{x}}{\partial{u}^{k}} \cdot \frac{\partial \mathbf{n}}{\partial{u}^{j}}} $$
\\\\
$${=\sum_{j} \frac{\partial \eta^{j}}{\partial{u}^{j}}+\sum_{j} \eta^{j} \frac{1}{2} \mathbf{tr}\left({G}^{-1} \frac{\partial G}{\partial {u}^{j}}\right)+\sum_{j, k} \phi{g}^{j k} \frac{\partial \mathbf{x}}{\partial {u}^{k}} \cdot \frac{\partial \mathbf{n}}{\partial{u}^{j}}}$$  \\\\
$$=\sum_{j} \frac{1}{\sqrt{g}} \frac{\partial\left(\eta^{j} \sqrt{g}\right)}{\partial u^{j}}+\sum_{j, k} \phi g^{j k} \frac{\partial \mathbf{x}}{\partial u^{k}} \cdot \frac{\partial \mathbf{n}}{\partial u^{j}}$$ \\\\




$$
{=\operatorname{div}_{\mathbf{\Gamma}} {\eta}^{T}-{\phi}{H}}
$$

So we have \\
$$
\left.\frac{d}{d t} A\left(\Phi_{t}(\Gamma)\right)\right|_{t=0}=\int_{\Gamma}\left\{\operatorname{div}_{r} \boldsymbol{\eta}^{T}-H \boldsymbol{\eta} \cdot \mathbf{n}\right\} d \boldsymbol{A}
 $$ \\




Now since $\Gamma$ is closed and has no boundary Therefore, $$\int_{\Gamma}div_{\Gamma}\eta^{T}d\textbf{A} = 0$$
so \\\\
\begin{equation}
\label{eq33}
    \left.\frac{d}{d t} A\left(\Phi_{t}(\Gamma)\right)|_{t=0}=\int_{\Gamma}\left\{\operatorname{div}_{r} \boldsymbol{\eta}^{T}-H \boldsymbol{\eta} \cdot \mathbf{n}\right\} d \boldsymbol{A}= -\int_{\Gamma}H \boldsymbol{\eta} \cdot \mathbf{n}\right\} d \boldsymbol{A}   \\\\
    \end{equation}
\end{proof}
\end{theorem}
 \subsection{$C^{k}$ Isoperimetric problem} 
 \label{ss:9}
    Let $\Omega$ be a bounded domain in $R^{n}$, with $C^{k}$ boundary, $k \geq 1 .$ We
say that $\Omega$ is a solution to the $C^{k}$ isoperimetric problem if, for any domain $D$
with $C^{k}$ boundary and volume equal to that of $\Omega,$ we have $A(\partial D) \geq A(\partial \Omega) .$
\\




We say that $\Omega$ is a
$C^{k}$ extremal of the isoperimetric functional if, for any 1 -parameter family of
$C^{k}$ diffeomorphisms $\Phi_{t}: \mathbf{R}^{n} \rightarrow \mathbf{R}^{n}$ satisfying \par $V\left(\Phi_{t}(\Omega)\right)=V(\Omega)$ for all $t,$ we
have \\\\
\begin{equation}
\label{eq34}
\left.\frac{d}{d t} A\left(\Phi_{t}(\partial \Omega)\right)\right|_{t=0}=0
\end{equation}
\\\\
\begin{theorem}

  Assume that $\Omega$ is a solution to the $C^{2}$ isoperimetric problem,
with volume of $\Omega$ equal that of the unit $n$ -disk in $\mathbb{R}^{n}$. Then the mean curvature
H of $\Gamma$ satisfies
$$
-H \leq n-1
$$
on all of $\Gamma$.
\end{theorem}

\begin{proof}

Consider the Isoperimetric functional \\\\
\begin{equation}
\label{eq35}
J(D)=\frac{A(\partial D)}{V(D)^{1-1 / n}}
 \end{equation}
 \\\\
where D varies over bounded domains in $R^{n}$ having $C^{2}$ boundary.
As in above \\ theorem We have $\phi_{t}$, 1-parameter family of diffeomorphism of $\mathbb{R}^{n},$ with corresponding time-dependent vector field $$\boldsymbol{X} = \boldsymbol{X}(x, t)$$ and at time zero,\\ $$\boldsymbol{\xi}(\boldsymbol{x})=\boldsymbol{X}(\boldsymbol{x}, \boldsymbol{0})$$ and  restircting on its boundary,   \\      $$\boldsymbol{\eta}=\boldsymbol{\xi} | \Gamma $$ \\
As $\omega$ is a solution to the  $C^{2}$ Isoperimetric problem, $\Omega$ will minimize the isoperimetric functional $J(D)$ so we have


$$\left.\frac{d}{d t} J\left(\Phi_{t}(\Omega)\right)\right|_{t=0} = 0 $$ \\ 
Now on differenciating $J(\Phi_{t}(\Omega))$ with respect to t,
we get \\\\
$$\left.\frac{d}{d t} J\left(\Phi_{t}(\Omega)\right)\right|_{t=0} =  -\frac{1}{V(\Omega)^{1-1 / n}} \int_{\Gamma} H \eta \cdot \mathbf{n} d A+\left(\frac{1}{n}-1\right) \frac{A(\Gamma)}{V(\Omega)^{2-1 / n}} \int_{\Gamma} \eta \cdot \mathbf{n} d A 
$$ \\\\
which implies \\\\
$$
-\frac{\int_{\Gamma} H \eta \cdot \mathbf{n} d A}{\int_{\Gamma} \eta \cdot \mathbf{n} d A}=\frac{n-1}{n} \frac{A(\Gamma)}{V(\Omega)}
$$
Now $\Omega$ has its $V(\Omega) = \omega_{n}$ we should have $A(\Gamma) \leq c_{n-1}$ since $\Omega $ is solution of Isoperimetric Problem \\
 recall that \\ $$ \omega_{\mathbf{n}}=\frac{\mathbf{c}_{\mathbf{n}-1}}{n}$$ \\
 together  implies \\\\
 \begin{equation}
 \label{eq36}
 -\frac{\int_{\Gamma} H \eta \cdot \mathbf{n} d \boldsymbol{A}}{\int_{\Gamma} \boldsymbol{\eta} \cdot \mathbf{n} d \boldsymbol{A}} \leq \boldsymbol{n}-1 
 \end{equation}
 let $\phi$ be a nonnegative $C^{\infty}$ function for any $w_{0} \in \Gamma,$ compactly supported on a neighborhood of $w_{0}$ in $\mathbb{R}^{n}$.
 Choosing a perticular $X(x, t)$ which is time-independent to  simplify above expression, \\
$$
X(x, t)=\phi(x) \mathbf{n}_{u_{0}}
$$
\\
and now picking $\phi$ with sufficiently small support about $w_{o}$ we get left hand side of expression to be 

$ -H\left(w_{0}\right )$ \\

  and that finally gives
  \\
   $$ -H(w_{0})\leq n-1 $$ \\
    for any $ \omega_{0} \in \Gamma$ hence it is proved on all of $\Gamma$ 
\end{proof}





\subsection{Theorem}
Assume $\Omega$ is a bounded domain in $\mathbf{R}^{n}$, with $C^{2}$ boundary $\Gamma$
and $\mathbf{n}$, its exterior normal unit vector field along $\Gamma$. Assume the mean curvature
$H$ of $\Gamma$ satisfes
$$
-H \leq n-1
$$
along all of $\Gamma$. Then
$$
A(\Gamma) \geq \mathfrak{c}_{\mathfrak{n}-1}
$$
with equality iff $\Omega$ is an disk in $\mathbb{R}^{n}$ \\\\
\textbf{Proof:} \ \textbf{Edit karna hai decide whether to write or not }\\

 Let $\Omega$ with its boundary,  $\bar{\Omega} =\Omega \cup \Gamma,$ and denote its convex hull  $\mathcal{C}$  and denote the boundary of $\mathcal{C}$  to $\Sigma=\partial \mathcal{C}$ 

Let $w \in \Sigma \backslash(\Sigma \cap \Gamma),$ T asupporting hyperplane of $\mathcal{C}$ at $w$. (See Figure II. $1.1 .$ ) Then there exists an $x \in \Pi \cap(\Sigma \cap \Gamma)$ (if not, then $\mathcal{C}$ would not be the smallest convex set containing $\bar{\Omega}$ ), which implies (a) $\Pi$ is the tangent hyperplane to $\Gamma$
at $x,$ and $(b)$ the line segment $w x$ is contained in $\Pi$. These two imply that $\Pi$ is the unique supporting hyperplane to $\mathcal{C}$ at every point of $u x,$ except possibly at
w. If $\Pi$ were not the unique supporting hyperplane of $\mathcal{C}$ at $w$, then $\Sigma$ would have a conical singularity at $w,$ which would imply that $\mathcal{C}$ is not the minimal convex set containing $\bar{\Omega}$. Therefore, every point of $\Sigma$ has a unique supporting hyperplane, that is, $\boldsymbol{\Sigma}$ is everywhere differentiable.



\subsection{Theorem}
 Let $\Omega$ be a bounded domain in $\mathbb{R}^{n}$ with $C^{2}$ boundary, that is an extremal of the $C^{2}$ isoperimetric functional. Then  $\partial \Omega$ has constant mean curvature.

\textbf{Proof: }  

We denote 
$\phi = \eta\cdot{n}$,
by first part of the above theorem \\ 

and by second part of above theorem 
$$\int_{\Gamma} \phi H d A=0$$

Thus we have \\

\begin{center}
   $ \int_{\Gamma} \phi H d A=0 \forall \phi \text { such that } \int_{\Gamma} \phi d A=0$ \\ 
\end{center}
from here H will be a constant 

\chapter{Isoperimetric Inequality in Convex Subsets of $\mathbf{R}^{n}$ }
  \label{chap:c4}


% may need to change topic of this subsection to convex geometry 








































\section{Convex Sets and its properties } \label{s:1}
we take structure of  $\mathbb{R}^n$ as a vector space 
For $ A \subset \mathbb{R}^n$ we call A, a \textbf{Convex} set  if for $\alpha \in [0,1]$ we have $\alpha x +(1- \alpha)y \in A $ for each  $x, y \in A.$  i.e. line segment joining x and y in A belongs inside A.\\
\begin{itemize}
    \item  For example open and closed line segments joining x and y in $\mathbb{R}^n$ are convex sets in $\mathbb{R}^n$

\item  

 Ball of radius $r \geq 0$,  $B (r):=\left\{x \in \mathbb{R}^{n}:\|x\| \leq r\right\}$,  and its translates
 
 \end{itemize}

 For two points $x,y \in A$ and for $\alpha \in [0,1]$, $\alpha x +(1- \alpha)y  $ is called \textbf{Convex combination} of x and y and its generalisation for any number of  points is following. \\

Let $k \in \mathbb{N},$ let $x_{1}, \ldots, x_{k} \in \mathbb{R}^{n},$ and let $\alpha_{1}, \ldots, \alpha_{k} \in[0,1]$ with $\alpha_{1}+\ldots+\alpha_{k}=1,$ then
$\alpha_{1} x_{1}+\cdots+\alpha_{k} x_{k}$ is called a \textbf{Convex combination} of the points $x_{1}, \ldots, x_{k}$

\begin{theorem}
\label{t:4}
$A$ set $A \subset \mathbb{R}^{n}$ is convex, iff  all convex combinations of points in A lie
in $A$.
\end{theorem}

\begin{proof}

 Assume $A$ is convex and $k \in \mathbb{N} .$ We will be using induction on $k$ \\ for $k=1,$ induction step is trivial  . For the step from $k-1$ to $k, k \geq 2,$ assume $x_{1}, \ldots, x_{k} \in A$ and $\alpha_{1}, \ldots, \alpha_{k} \in[0,1]$ with $\alpha_{1}+\ldots+\alpha_{k}=1 .$ We may assume $\alpha_{k}$ is not 0 or 1.\\ and define 
$$
\beta_{i}:=\frac{\alpha_{i}}{1-\alpha_{k}}, \quad i=1, \ldots, k-1
$$
hence $\beta_{i} \in[0,1]$ and $\beta_{1}+\ldots+\beta_{k-1}=1 .$ By the induction hypothesis, $\beta_{1} x_{1}+\ldots+\beta_{k-1} x_{k-1} \in A$
and by the convexity
$$
\sum_{i=1}^{k} \alpha_{i} x_{i}=\left(1-\alpha_{k}\right)\left(\sum_{i=1}^{k-1} \beta_{i} x_{i}\right)+\alpha_{k} x_{k} \in A
$$
hence done for first side\\
for the other direction, if all convex combinations of points in A lie in A then on just taking $k =2$ condition for convex set will be satisfied.\\

\end{proof}.

\begin{mydef} 


For a arbitrary family  $\left\{A_{i}: i \in I\right\}$ of convex sets in   $\mathbb{R}^{n}$, its intersection $\bigcap_{i \in I} A_{i}$ is also a convex set in $\mathbb{R}^{n}$  for a given set $A \subset \mathbb{R}^{n}$.\\ Intersection of all convex sets containing $A$ is called its  \textbf{Convex Hull} and is denoted by  \textbf{conv A}.\\
 Intuitively Convex hull of $A$  is the smallest convex set containing $A$ which fills its non convex parts and turns out to be set of all convex combinations of points of $A$ which is following theorem.
 
 \end{mydef}
 
 

\begin{theorem}
\label{t:5}
For  any $A \subset \mathbb{R}^n$
 $$ \operatorname{conv} A=\left\{\sum_{i=1}^{k} \alpha_{i} x_{i}: k \in \mathbb{N}, x_{1}, \ldots, x_{k} \in A, \alpha_{1}, \ldots, \alpha_{k} \in[0,1], \sum_{i=1}^{k} \alpha_{i}=1\right\} $$

 \begin{proof}
 Let us denote $$ B = \left\{\sum_{i=1}^{k} \alpha_{i} x_{i}: k \in \mathbb{N}, x_{1}, \ldots, x_{k} \in A, \alpha_{1}, \ldots, \alpha_{k} \in[0,1], \sum_{i=1}^{k} \alpha_{i}=1\right\} $$  which is expression on the right hand side of theorem.\\\\
 \textbf{(i)}  $ \mathbf{B} \subset \mathbf{\operatorname{conv} A}  $\\\\
     For any arbitrary convex set $C \supset A$, on considering previous Theorem 3.1.1 for convex set C, all convex combinations of points in C should lie in C so all convex combinations of points of A (subset of  C) will definitely lie inside C. \\ 
 So we would have $B\subset C $ for all such C and now since  $\operatorname{conv} A $ is intersection of all such convex sets like C containing A we will have  $ B \subset \operatorname{conv} A.  $  
 
 \textbf{(ii)}  $ \mathbf{B} \supset \mathbf{\operatorname{conv} A}  $\\\\
 We claim that B is convex set since for any two elements $\alpha_{1} x_{1}+\cdots+\alpha_{k} x_{k}$ and \\ $\gamma_{1} y_{1}+\cdots+\gamma_{m} y_{m}$  of B \\
 we have 
 
 
\begin{multline}
         \beta\left(\alpha_{1} x_{1}\right)  \left(\cdots+\alpha_{k} x_{k} \right) +(1-\beta)\left(\gamma_{1} y_{1}+\cdots+\gamma_{m} y_{m}\right) \\ = \beta \alpha_{1} x_{1}+\cdots+\beta \alpha_{k} x_{k}+(1-\beta) \gamma_{1} y_{1}+\cdots+(1-\beta) \gamma_{m} y_{m}  
\end{multline}
 
 where  $x_{i}, y_{j} \in A$ and coefficients $\beta, \alpha_{i}, \gamma_{j} \in[0,1]$ with  condition $\alpha_{1}+\ldots+\alpha_{k}=1$ and $\gamma_{1}+\ldots+\gamma_{m}=1$ 

hence we have \\
$
\beta \alpha_{1}+\ldots+\beta \alpha_{k}+(1-\beta) \gamma_{1}+\cdots+(1-\beta) \gamma_{m}=\beta+(1-\beta)=1
$
Now since $ B \supset A $ and B is convex set and $\operatorname{c}$
 
we have  $ \mathbf{B} \supset \mathbf{\operatorname{conv} A}  $ \par
 Hence from (i) and (ii) we are done. \\
 We also have $A = \operatorname{conv} A$ iff A is a convex set as smallest convex set containing A will be itself A. \\

 \end{proof}
 
\end{theorem}

\begin{mydef}
We define linear combination $\alpha A+\beta B$ of two sets A and B $\in \mathbb{R}^n$ to be 
$\alpha A+\beta B:=\{\alpha x+\beta y: x \in A, y \in B\}$ where $\alpha,\beta \in \mathbb{R}$.
This combination is also called Minkowski addition.


\begin{itemize}
    \item $\alpha A+\beta B $ is also convex if A and B are. 
\item
Generally for any set A, $A +A = 2A $ and A - A = 0  does not hold but in case when A is a convex set it holds and 
 for $\alpha, \beta \geq 0,$ we also have $\alpha A+\beta A=(\alpha+\beta) A,$ because of convex property. \end{itemize}
 \end{mydef}
 
 
 
 

 
 
 
 
 
 
 
 
 
 
 
 
 
 
 
 
 
 
 
 
 
 
 
 
 
 
 
\subsection{ Convex Polytopes and Polyhedral sets }
Convex polyhedron and convex polytopes are very interesting class of objects 
\begin{mydef}

Intersection of finitely many closed half -spaces  is called \\
\textbf{ Convex Polyhedral Set  or Polyhedron}
 where half-spaces are defined as $\{ x : a^{T}x \leq b\} $  where a is non-zero vector in $\mathbb{R}^n$ and b is another vector in  $\mathbb{R}^n.$  
 
 
\end{mydef}


\begin{itemize}
\item
 Polyhedral sets are closed as they are intersection of closed half-spaces. 
  \item
 Polyhedral sets  are convex sets  also because intersection of convex sets is convex we just need to show that half-spaces are convex.\par
   for $a^{T} x_{1} \leq b, a^{T} x_{2} \leq b$ \par 
 we have 
 
 $a^{T}\left(\alpha +(1- \alpha) x_{2}\right)=\alpha a^{T} x_{1}+(1-\alpha) a^{T} x_{2} \leq b$ 
  for $x_{1} $ and $x_{2}$ in half space.  \par
  So half-spaces are convex hence  Convex Polyhedron  is convex.\par 
  \item
   Convex polyhedron may not be bounded, just one half space can be a convex polyhedron which is ofcourse unbounded 
   \end{itemize}
   
  \begin{mydef}
  
  Convex hull of finitely many points $x_1,\ldots ,x_k \in \mathbb{R}^n $ is called \textbf{Convex Polytope.} 
 These are bounded in-fact every bounded convex polyhedron  is a convex polytope.\\
  Every convex polytope is a polyhedron but not in the other side as convex  polytope need to be bounded, once bounded both are equivalent.
 
 
  \end{mydef}
  

\begin{itemize}
\item

Triangle is convex hull of 3 distinct points in space and so is a convex polytope polytope as well as a convex polyhedron as it is intersection of 3 closed half-spaces.

\item
  convex hull of 4 distinct points in space is tetrahedron, and of vertices of cube is cube itself 
 
 \item
 For a polytope P its \textbf{Vertex} is defined as point $x \in P$ for which $P \backslash \{x\} $ is  still convex and P is convex combination of its vertices which is  the next theorem. 
 \end{itemize}
\begin{mydef}

 Points $ x_1, \ldots ,x_k \in \mathbb{R}^n $ are called \textbf{Affinely Independent} if vectors \par
 $x_{2} - x_{1}, \ldots  x_{k} - x_{1} $ are linearly independent.
 \end{mydef}  
 \begin{itemize}
 \item Linearly Independent is defined for vectors in $\mathbb{R}^n $ whereas Affinely Independent is defined for points in $\mathbb{R}^n $
 \end{itemize}


 



 

 
 
 
 
 
 




\begin{mydef}	
 A \textbf{Simplex} is the the convex hull of affinely independent points and \par 
  an \textbf{r-Simplex} is the convex hull of $r+ 1$ affinely independent points.
  \\\\
  \end{mydef} 


\begin{theorem}
\label{t:6}
Let $P$ be a polytope in $\mathbb{R}^{n},$ and let $x_{1}, \ldots, x_{k} \in \mathbb{R}^{n}$ be distinct points.\par
(a) If $P=\operatorname{conv}\left\{x_{1}, \ldots, x_{k}\right\},$ then $x_{1}$ is a vertex of $P,$ if and only if $x_{1} \notin \operatorname{conv}\left\{x_{2}, \ldots, x_{k}\right\}$ \par
(b) $P$ is the convex hull of its vertices. \\\\

\begin{proof}

\textbf{(a)}  \par 


 If  $x_{1}$ is a vertex of $P $ then  from definition of vertex, $P \backslash\left\{x_{1}\right\}$ will be convex and $x_{1} \notin P \backslash\left\{x_{1}\right\} .$ Hence we have  $\operatorname{conv}\left\{x_{2}, \ldots, x_{k}\right\} \subset P \backslash\left\{x_{1}\right\},$ 
 \par 
 so  $x_{1} \notin \operatorname{conv}\left\{x_{2}, \ldots, x_{k}\right\}$ \\\\
For the other direction, on assuming that $x_{1} \notin \operatorname{conv}\left\{x_{2}, \ldots, x_{k}\right\} .$ and provided that  $x_{1}$ is not a vertex of $P,$ there exist distinct points $a, b \in P \backslash\left\{x_{1}\right\}$ and $\lambda \in(0,1)$ such that $x_{1}=(1-\lambda) a+\lambda b .$ 

As  $P=\operatorname{conv}\left\{x_{1}, \ldots, x_{k}\right\},$ a and b can be written as convex linear combination of $x_{1},\ldots ,x_{k} $   so there should exist $k \in \mathbb{N}, \mu_{1}, \ldots, \mu_{k} \in[0,1]$ and $\tau_{1}, \ldots, \tau_{k} \in[0,1]$ with $\mu_{1}+\ldots+\mu_{k}=1$ and $\tau_{1}+\ldots+\tau_{k}=1$
such that $\mu_{1}, \tau_{1} \neq 1$ and \par
\begin{equation}
\label{eq37}
a=\sum_{i=1}^{k} \mu_{i} x_{i}, \quad b=\sum_{i=1}^{k} \tau_{i} x_{i}
\end{equation}  
as $x_{1}=(1-\lambda) a+\lambda b .$ we get on putting values of a and b 

\begin{equation} 
\label{eq38}
x_{1}=\sum_{i=1}^{k}\left((1-\lambda) \mu_{i}+\lambda \tau_{i}\right) x_{i}
\end{equation}
that finally gives 
\begin{equation} 
\label{eq39}
x_{1}=\sum_{i=2}^{k} \frac{(1-\lambda) \mu_{i}+\lambda \tau_{i}}{1-(1-\lambda) \mu_{1}-\lambda \tau_{1}} x_{i}
\end{equation}

where $(1-\lambda) \mu_{1}+\lambda \tau_{1} \neq 1$ \par
 so $x_{1}$ is written as a convex combination of $x_{2}, \ldots, x_{k}$ in last equation, which is  a contradiction.
\\\\\\
\textbf{(b)}  For $P=\operatorname{conv}\left\{x_{1}, \ldots, x_{k}\right\}$ we can remove points  which are not vertex  
one by one and this will not change convex hull and will be equal to P as removed points can be written as convex combination of other remaining vertex points. \\
If $x$ is a vertex of P but $x \notin \left\{x_{1}, \ldots, x_{k}\right\} $ then P would be,
$P=\operatorname{conv}\left\{x,x_{1}, \ldots, x_{k}\right\}$ 
which implies that x can not be written as convex combination of $x_{1}, \ldots x_{k}$ \\ i.e. 
 $x \notin  \operatorname{conv} \left\{x_{1}, \ldots, x_{k}\right\} = P  $ so $x \notin P$ that gives contradiction.\par
 Hence P is the convex hull of its vertices.
\end{proof}


\end{theorem}


\begin{mydef}
 \textbf{The support function }$h_{A}: \mathbb{R}^{n} \rightarrow(-\infty, \infty) $  for a nonempty and convex $A \subset \mathbb{R}^{n}$ is defined as
$$
h_{A}(u):=\sup _{x \in A}\langle x, u\rangle, \quad u \in \mathbb{R}^{n}
$$

\end{mydef}



\subsection{Convex Function}



\begin{mydef}. 
For a function   $f: \mathbb{R}^{n} \rightarrow(-\infty, \infty],$ first we define $\text { epi } f, $  
\begin{equation}
\label{eq40}
\text { epi } f:=\left\{(x, \alpha): x \in \mathbb{R}^{n}, \alpha \in \mathbb{R}, f(x) \leq \alpha\right\} \subset \mathbb{R}^{n} \times \mathbb{R} 
\end{equation}
\end{mydef}
  then f is a convex function if epi $f$ is a convex subset of $\mathbb{R}^{n} \times \mathbb{R}=\mathbb{R}^{n+1}$  and \\  f is concave, if $-f$ is convex. 
  
  %Thus, for a convex function $f$ we exclude the value $-\infty,$ whereas for a concave function we exclude $\infty
 If $A \subset \mathbb{R}^{n}$ is a subset, a function $f: A \rightarrow(-\infty, \infty)$ is called convex, if the extended function $\tilde{f}: \mathbb{R}^{n} \rightarrow(-\infty, \infty],$ given by
$$
\tilde{f}:=\left\{\begin{array}{ll}
f & \text { on } \quad A \\
\infty & \mathbb{R}^{n} \backslash A
\end{array}\right.
$$
is convex,  Where  $A$ is a convex set. wlog we can assume that convex functions are always defined on all of $\mathbb{R}^{n}$ with this construction.\\
convex functions $f: \mathbb{R}^{n} \rightarrow(-\infty, \infty]$ at points, where $f$ is finite. \\
We define effective domain of the function where function is finite
$$
\operatorname{dom} f:=\left\{x \in \mathbb{R}^{n}: f(x)<\infty\right\}
$$



\begin{mydef}
 \textbf{The support function }$h_{A}: \mathbb{R}^{n} \rightarrow(-\infty, \infty) $  for a nonempty and convex $A \subset \mathbb{R}^{n}$ is defined as
$$
h_{A}(u):=\sup _{x \in A}\langle x, u\rangle, \quad u \in \mathbb{R}^{n}
$$

\end{mydef}




\begin{theorem}
\label{t:7}
 A function $f: \mathbb{R}^{n} \rightarrow(-\infty, \infty]$ is convex, if and only if
$$
f(\alpha x+(1-\alpha) y) \leq \alpha f(x)+(1-\alpha) f(y)
$$
for all $x, y \in \mathbb{R}^{n}, \alpha \in[0,1]$
\begin{proof}

Since $f$ is convex, we have epi $f=\{(x, \beta): f(x) \leq \beta\}$ is convex, from definition of convex set, we have following. \\
whenever $\left(x_{1}, \beta_{1}\right),\left(x_{2}, \beta_{2}\right) \in$ epi $f,$ i.e. $f\left(x_{1}\right) \leq \beta_{1}, f\left(x_{2}\right) \leq \beta_{2}$, for all $\alpha \in[0,1]$
 We have
 $\alpha\left(x_{1}, \beta_{1}\right)+(1-\alpha)\left(x_{2}, \beta_{2}\right) \in \text{epi} f$
 $$
\alpha\left(x_{1}, \beta_{1}\right)+(1-\alpha)\left(x_{2}, \beta_{2}\right)=  \left(\alpha x_{1}+(1-\alpha) x_{2}, \alpha \beta_{1}+(1-\alpha) \beta_{2}\right)
$$
and  $ \alpha x_{1}+(1-\alpha) x_{2}, \alpha \beta_{1}+(1-\alpha) \beta_{2} \in \text { epi } f $
\\
So $f$ is convex, iff
$$
f\left(\alpha x_{1}+(1-\alpha) x_{2}\right) \leq \alpha \beta_{1}+(1-\alpha) \beta_{2}
$$
for   all $ \beta_{1} \geq f\left(x_{1}\right), \beta_{2} \geq f\left(x_{2}\right) $ and  all $ x_{1}, x_{2} \in \mathbb{R}^{n}, \alpha \in[0,1]$ \\
as it is satisfied for all $ \beta_{1} \geq f\left(x_{1}\right), \beta_{2} \geq f\left(x_{2}\right) $, it will also satisfy for $\beta_{1}=f\left(x_{1}\right), \beta_{2}=f\left(x_{2}\right),$  hence proved.  
\end{proof}
\end{theorem} 

We can  generalize above theorem as following \\
$f$ is convex, if and only if
\begin{equation}
\label{eq41}
f\left(\alpha_{1} x_{1}+\cdots+\alpha_{k} x_{k}\right) \leq \alpha_{1} f\left(x_{1}\right)+\cdots+\alpha_{k} f\left(x_{k}\right)
\end{equation}
for all $k \in \mathbb{N}, x_{i} \in \mathbb{R}^{n},$ and $\alpha_{i} \in[0,1]$ with $\sum \alpha_{i}=1$



\begin{mydef}
 A convex function $f: \mathbb{R}^{n} \rightarrow(-\infty, \infty]$ is closed, if epi $f$ is closed.
for convex  $f: \mathbb{R}^{n} \rightarrow(-\infty, \infty]$   cl epi $f$ is the closer of set epigraph of $f$ and hence epigraph of a closed convex function, which we denote by cl $f$
\end{mydef}

\begin{theorem}	

 On int dom $f$, convex function $f: \mathbb{R}^{n} \rightarrow(-\infty, \infty]$ is continuous and on compact subsets of int dom $f$,it is Lipschitz continuous 	

\end{theorem}	
\newpage	
Now we examine differentiability characteristics of convex functions	
\begin{theorem}	

 Let $f: \mathbb{R}^{1} \rightarrow(-\infty, \infty]$ is convex function.	
\begin{enumerate}	
    \item 	
 In each point $x \in$ int dom $f$, the right derivative $| f^{+}(x)$ and the left derivative $f^{-}(x)$ exist and fulfill $f^{-}(x) \leq | f^{+}(x)$	
\\	
\item	
 On int $\operatorname{dom} f,$ the functions $f^{+}$ and $f^{-}$ are monotonically increasing and, for almost all	
$x \in$ int dom $f\left(\text { with respect to the Lebesgue measure } \lambda_{1} \text { on } \mathbb{R}^{1}\right),$ we have $f^{-}(x)=f^{+}(x),$ hence $f$ is almost everywhere differentiable on cl dom $f$	
\\	
\item	
 Moreover, $f^{+}(x)$	
is continuous from the right and $f^{-}$ is continuous from the left, and $f$ is the indefinite integral of $ f^{+}, \text { (of } f^{-} \text {and of } f^{\prime} in \operator {int dom} f$	
\end{enumerate}	






\end{theorem}








%comment is here include  it if you want or if there is an y use of it after 
\begin{comment}
\begin{mydef}
A function $f: \mathbb{R}^{n} \rightarrow(-\infty, \infty]$ is \textbf{positively homogeneous} of degree 1 if
$$
f(\alpha x)=\alpha f(x), \quad \text { for all } x \in \mathbb{R}^{n}, \alpha \geq 0
$$
$f$ is convex if and only if it is subadditive, given  $f$ is positively homogeneous  as
$$
 f(x+y) \leq f(x)+f(y)  \text{and }f(\alpha x)=\alpha f(x) $$,
  \quad \text { for all } R$ gives definition of convex function
\end{mydef}
\end{comment}
%comment is here include  it if you want or if there is an y use of it after 
 \subsection{Convex Body}
 
\textbf{Convex Body } in  $\mathbb{R}^n$  is a non empty compact convex sets with non-empty interior.\\
e.g.\\
\begin{itemize}
    \item   Convex Polytope in $\mathbb{R}^n$ are Convex Body but polyhedron is not as it need not be compact because it may be unbounded. 
    
    i.e.\\
    
    \item 
Triangle in plane is convex body as it has non-empty interior, however it is not a convex body if is considered as topological subset in space  $\mathbb{R}^3$ as then it will have  empty interior. 
\\  
But often we don't take non-empty interior condition seriously and can be left and can also consider all lower dimension object imbedded in higher dimension space.
\end{itemize}
 
Convex bodies can also be defined in following way.\\
     A Region  bounded by a hyper-surface with non-negative definite second fundamental form (i.e. $\mathbb{I} \geq 0 $ ) is a convex body. this condition is local but  applies global convexity. \\
     $\mathbb{I} \geq 0 $ is necessary, we can see it intuitionally in  $\mathbb{R}^3$ as all points must be elliptic if any point is hyperbolic then it will not imply global convexity. \\
     The boundary of a convex body is $C^{2}$ almost everywhere   \\\\
     
     
     % need to edited try to complete proof and define it correctly other wise don't give this def
 
 
 
 
 
 
 
 
 We denote  the space of convex bodies by  $\mathcal{K}^{n}$. For  $\mathcal{K}^{n}$  we are taking convex bodies which have empty interior i.e. lower-dimensional bodies are also included in $\mathcal{K}^{n}$. 
 Sum of two convex bodies is a convex body. \\
 $$
   K, L \in \mathcal{K}^{n} \Longrightarrow K+L \in \mathcal{K}^{n}
$$ 
 So  set $\mathcal{K}^{n}$ is closed under addition,
 Also we have
 $$ 
K \in \mathcal{K}^{n}, \alpha \geq 0 \Longrightarrow \alpha K \in \mathcal{K}^{n}
$$
So  $\mathcal{K}^{n}$ is closed under scale multiplication \\
Also, since the reflection $-K$ of a convex body $K$ is again a convex body, $\alpha K \in \mathcal{K}^{n}$, for all $\alpha \in \mathbb{R}$, and hence it satisfies properties of being a cone so,  $\mathcal{K}^{n}$ is a convex cone and 
%More things to added after this. now started writing convex function instead completing this 
\begin{mydef}	
For  $ K, L \in \mathcal{K}^{n}, $, We define distance between $k$ and $L$ as 
\begin{equation}
\label{eq42}
d(K, L):=\inf \{\varepsilon \geq 0: K \subset L+B(\varepsilon), L \subset K+B(\varepsilon)\}
\end{equation}
\end{mydef}
\quad 
\\\\
We will use results of following theorem without proof 
\begin{theorem}
\label{t:8}
For  $K \in \mathcal{K}^{n}$ and $\varepsilon>0$ \\
\begin{enumerate}
\item
There exist  polytope $ P  \in \mathcal{P}^{n}$ such that  $P \subset K$ and  $d(K, P) \leq \varepsilon$ \newline 
and \newline
  there exist polytope $Q \in \mathcal{P}^{n}$such that  $K \subset Q$ and $d(K, Q) \leq \varepsilon$ 
\item 
  There exists a polytope $P \in \mathcal{P}^{n}$ with $P \subset K \subset(1+\varepsilon) P$ \\
  if $0 \in \text{rel int } K $ \label{4.1.5}
\end{enumerate}
\end{theorem}
 \subsection{Volume and surface area of convex body}
We can define Volume and surface area for convex body in elementary sense as it is convex. By the way its Volume can also be defines as lebesgue measure $\lambda_{n}(K)$ of K  for $ K \in  \mathcal{K}^{n} $. \\
In elementary sense, first Volume and surface area for polytopes are defined recursively on dimension n and then by approximation they are defined for  arbitrary convex bodies. \newline
\textbf{Remark-} 
For a convex body  K, its support set $K(u),u \in S^{n-1} $ lies in a hyperplane parallel to 
$u^{\perp}$ and on translating $K(u)$ we get its orthogonal projection $K(u) | u^{\perp}$, and we think  $K(u) | u^{\perp}$ as a convex body in $\mathbb{R}^{n-1}$  on identifing $u^{\perp}$ with $\mathbb{R}^{n-1}$. \\
  For determimnig volume of convex body in $\mathbb{R}^{n}$ recursively, we assume that we already  know volume in $(n-1)$ - dimension and we denote $V^{(n-1)}\left(K(u) | u^{\perp}\right)$, the $(n-1)$ - dimensional volume of this projection as this projection is a convex body in $\mathbb{R}^{n-1}$ 
\\
Now lets define volume recursively for polytope first 
  \begin{mydef}
For a polytope $P \in \mathcal{P}^{n}$,\\
if  $n=1,$we have  $P=[a, b]$ where $a \leq b,$ then we define volume $V^{(1)}(P):=b-a$ and  surface area $A^{(1)}(P):=2$ 
\\\\
  For  $ n \geq 2  $ and $\operatorname{dim} P \leq n-2$, there is no facets in the polytope $P$, hence its volume $V(P)=0$ and surface area $A(P)=0$ \\
  
 We define set $A_{{u}}$ to be the set of all $u \in \mathbb{S}^{n-1}$ for which $P(u)$ is a facet of P. Then 
\begin{equation}
\label{eq43}
V^{(n)}(P):=\left\{\begin{array}{ccc}
\frac{1}{n} $$ \sum_{u \in A_{u}} h_{P}(u) V^{(n-1)}\left(P(u) | u^{\perp}\right) & & \operatorname{dim} P \geq n-1 \\
0 & \text { if } & \operatorname{dim} P \leq n-2
\end{array}\right.
\end{equation}
and
\begin{equation}
\label{eq44}
A^{(n)}(P):=\left\{\begin{array}{ccc}
\sum_{u \in A_{u}} V^{(n-1)}\left(P(u) | u^{\perp}\right) & & \operatorname{dim} P \geq n-1 \\
0 & \text { if } & \operatorname{dim} P \leq n-2
\end{array}\right.
\end{equation}
Recall that here $h_{P}(u)$ is the support function of P at point u. \\
 When $\operatorname{dim} P=n-1, $ P is like a hyperplane of dimension $n - 1$  and hence there are only two support sets for $P$ which are its facets, $P=P\left(u_{0}\right)$ and $P=$ $P\left(-u_{0}\right),$ where $u_{0}$ is a normal vector to $P .$ In that case we have $$V^{(n-1)}\left(P\left(u_{0}\right) | u_{0}^{\perp}\right)=V^{(n-1)}\left(P\left(-u_{0}\right) | u_{0}^{\perp}\right)$$
 \newpage
and from the  definition of support function, we have $h_{P}\left(u_{0}\right)=-h_{P}\left(-u_{0}\right)$,\\ so we find $V(P)=0,$ which matches to lebesgue measure result also  and 
 
 \begin{equation}
 \label{eq45}
 A(P)=2 V^{(n-1)}\left(P\left(u_{0}\right) | u_{0}^{\perp}\right).\\\\
\end{equation}
\end{mydef} 
\begin{prop} \label{prop:1}
(Properties of the volume $V$ and surface area $A$ for polytopes $P, Q$ )
 \begin{enumerate}
 \item  $V(P)=\lambda_{n}(P)$
 \item  $V$ and $A$ are invariant with respect to rigid motions.
 \item $V(\alpha P)=\alpha^{n} V(P), A(\alpha P)=\alpha^{n-1} A(P),$ for $\alpha \geq 0$
 \item $V(P)=0,$ if and only if $\operatorname{dim} P \leq n-1$
 \item  if $P \subset Q,$ then $V(P) \leq V(Q)$ and $A(P) \leq A(Q)$    (Montotone Property)   \label{prop:5}
 \end{enumerate}
  \end{prop}
% if needed give proofs of these results %
 
Now we will be defining volume and surface area for a convex body.\\
 \begin{mydef}
  For a convex body $K \in \mathcal{K}^{n},$ we define
$$ V_{+}(K):=\inf _{P \supset K} V(P), \quad V_{-}(K):=\sup _{P \subset K} V(P)$$ 
and
$$ A_{+}(K):=\inf _{P \supset K} A(P), \quad A_{-}(K):=\sup _{P \subset K} A(P) $$ 
 \end{mydef}
 
\begin{theorem} 	
\label{t:9}
For $K \in \mathcal{K}^{n}$, We have 
\begin{equation}
\label{eq46}
 V_{+}(K)=V_{-}(K)
\end{equation}
   \begin{equation}
   \label{eq47}
   A_{+}(K)=A_{-}(K)
   \end{equation}
\end{theorem}         
\begin{proof}
We will be using Monotone property of Volume and surface area from above property \ref{prop:5} to prove it \\
$$V_{-}(K) \leq V_{+}(K)  $$  $$ A_{-}(K) \leq A_{+}(K)$$
We can assume that $0 \in \text{rel int} K$ as  $V_{-}(K), V_{+}(K), F_{-}(K)$ and $F_{+}(K)$ are \newline motion invariant so after a suitable transformation
 $0 \in \text{\text{\text{rel int}}} K$ 
 and from 2nd part of theorem 4.1.5,
for $\varepsilon>0,$ we can find a polytope P such that 
$$P \subset K \subset(1+\varepsilon) P .$$
\newpage
and then from above property \ref{prop:5} we have \\ 
\[
V(P) \leq V_{-}(K) \leq V_{+}(K) \leq V((1+\varepsilon) P)=(1+\varepsilon)^{n} V(P)
\] \\
using definitions $ V_{+}(K):=\inf _{P \supset K} V(P), \quad V_{-}(K):=\sup _{P \subset K} V(P)$ \\
and   \\
\[
A(P) \leq A_{-}(K) \leq A_{+}(K) \leq A((1+\varepsilon) P)=(1+\varepsilon)^{n-1} A(P)
\]  \\using definitions 
$ A_{+}(K):=\inf _{P \supset K} A(P), \quad A_{-}(K):=\sup _{P \subset K} A(P) $
 and \\ 
 
 now since $\varepsilon $ is arbitrary for $\varepsilon \rightarrow 0$ we have 
 
 $V_{+}(K)=V_{-}(K)$ and  $A_{+}(K)=A_{-}(K)$
 \end{proof}
 
 \begin{mydef}
 For $K \in \mathcal{K}^n$ its volume $V(K) $ and surface area $A(K)$ are defined as 
\[
  V(K) =: V_{+}(K)=V_{-}(K)
\]
$$\text{and} $$
\[
 A(K) :=  A_{+}(K) = A_{-}(K) 
\]
 
\end{mydef}
 \begin{prop}
 
For $ K \in \mathcal{K}^n $,  $V(K) $ and $A(K) $ has following properties just like proposition \ref{prop:1} for polytopes
\begin{enumerate}
    \item  $V(K)=\lambda_{n}(K)$
\item With respect to rigid motions $V$ and $A$ are invariant .
\item  for $\alpha \geq 0$, $V(\alpha K)=\alpha^{n} V(K), A(\alpha K)=\alpha^{n-1} A(K),$ 
\item $V(K)=0,$ iff $\operatorname{dim} K \leq n-1$
\item if $K \subset L,$ then  $A(K) \leq A(L)$ and $V(K) \leq V(L)$
\end{enumerate}
 \end{prop}
\begin{mydef}
We can also define $A(K)$ for $K \in \mathcal{K}^n$ in an alternative way in terms of derivative of $V(K)$ as following \\
\begin{equation}
\label{eq48}
A(K)=\lim _{\varepsilon \searrow 0} \frac{1}{\varepsilon}(V(K+B(\varepsilon))-V(K))
\end{equation}
\end{mydef}
\newpage
\subsection{Mixed Volume for Convex Body}
First we define mixed volume for polytopes recursively:  \\
We denote  $ \mathcal{N}(P_{1},P_{2},\ldots P_{k})$ to be the set of all facet normals of the convex polytope $P_{1}+P_{2}+ \ldots + P_{k}$
\begin{mydef}
We define the mixed volume $V^{(n)}\left(P_{1}, \ldots, P_{n}\right)$ of $P_{1}, \ldots, P_{n}$ recursively for polytopes $P_{1}, \ldots, P_{n} \in \mathcal{P}^{n},$ 
 
$$ \text{ for } n = 1, P_{1}=[a, b] \text { with } a \leq b,$$ 
    
  \begin{equation}
  \label{eq49}
      V^{(1)}\left(P_{1}\right):=V\left(P_{1}\right)=h_{P_{1}}(1)+h_{P_{1}}(-1) = b - a,  
  \end{equation} 
$$\text{and} \text { for } n \geq 2 $$ 
\begin{equation} \label{eq50}
\\
V^{(n)}\left(P_{1}, \ldots, P_{n}\right):=\frac{1}{n} \sum_{u \in \mathcal{N}\left(P_{1}, \ldots, P_{n-1}\right)} h_{P_{n}}(u) V^{(n-1)}\left(P_{1}(u)\left|u^{\perp}, \ldots, P_{n-1}(u)\right| u^{\perp}\right)
\end{equation}
  \end{mydef}     
 \begin{theorem}
 \label{t:10}
  If $\left(P_{1}^{(k)}\right)_{k \in \mathbb{N}}, \ldots,\left(P_{n}^{(k)}\right)_{k \in \mathbb{N}}$ are arbitrary approximating sequences\\
  
   converging to convex bodies  $K_{1}, \ldots, K_{n} \in \mathcal{K}^{n}$ respectively.
  \\
   i.e. each sequence of polytopes $P_{j}^{(k)}$ converges to a convex body, 
   $K_{j}, j=1, \ldots, n,$ as $k \rightarrow \infty,$ 
  then limit 
  \begin{equation}
  \label{eq51}
V\left(K_{1}, \ldots, K_{n}\right)=\lim _{k \rightarrow \infty} V\left(P_{1}^{(k)}, \ldots, P_{n}^{(k)}\right)
\end{equation}
exists and does not depend on the choice of approximating sequences $\left(P_{j}^{(k)}\right)_{k \in \mathbb{N}}. $ \newline
 Mixed volume of $K_{1}, \ldots, K_{n} $ is defined to be $V\left(K_{1}, \ldots, K_{n}\right)$ \newline
 
  and the mapping $V:\left(\mathcal{K}^{n}\right)^{n} \rightarrow \mathbb{R}$ defined $b y\left(K_{1}, \ldots, K_{n}\right) \mapsto V\left(K_{1}, \ldots, K_{n}\right)$ is called mixed volume.\newline
  
Precisely 
\begin{equation} 
\label{eq52}
V\left(K_{1}, \ldots, K_{n}\right)=\frac{1}{n !} \sum_{l=1}^{n}(-1)^{n+l} \sum_{1 \leq r_{1}<\cdots<r_{l} \leq n} V\left(K_{r_{1}}+\cdots+K_{r_{l}}\right)
\end{equation}
and, for $m \in \mathbb{N}, K_{1}, \ldots, K_{m} \in \mathcal{K}^{n}$ and $\alpha_{1}, \ldots, \alpha_{m} \geq 0$
 
 
\begin{equation}
\label{eq53}
V\left(\alpha_{1} K_{1}+\cdots+\alpha_{m} K_{m}\right)=\sum_{i_{1}=1}^{m} \cdots \sum_{i_{n}=1}^{m} \alpha_{i_{1}} \cdots \alpha_{i_{n}} V\left(K_{i_{1}}, \ldots, K_{i_{n}}\right)
\end{equation}
 \end{theorem}
 
 \subsubsection{Properties of Mixed Volume for Convex Body}
 
 for all $J, K, K_{1}, \ldots, K_{n} 
  \in \mathcal{K}^{n}$ \\ 
  \begin{enumerate}
      
  \item
$V(J, \ldots, J)=V(J)$ and $ nV(J, \ldots, J, B(1))=F(J)$ 
\item 
$V\left(K_{1}, \ldots, K_{n}\right)$ is symmetric in the indices $1,\ldots, n$
\item
 $V$ is multilinear, i.e.
 for all $ x$,$y$ $\geq 0 $
$$
V\left(x J+y K, K_{2}, \ldots, K_{n}\right)=x V\left(J, K_{2}, \ldots, K_{n}\right)+y V\left(K, K_{2}, \ldots, K_{n}\right)
$$
\item
for all $\alpha_{1}, \ldots, \alpha_{n} \in \mathbb{R}^{n}$
$$
V\left(K_{1}+\alpha_{1}, \ldots, K_{n}+\alpha_{n}\right)=V\left(K_{1}, \ldots, K_{n}\right)
$$
\item  
For all rigid motions $f$ 
$$ 
V\left(f K_{1}, \ldots, f K_{n}\right)=V\left(K_{1}, \ldots, K_{n}\right)
$$ 
\item
$V$ is continuous, i.e.
Whenever $K_{i}^{(j)} \rightarrow K_{i}, i=1, \ldots, n$
\[
V\left(K_{1}^{(j)}, \ldots, K_{n}^{(j)}\right) \rightarrow V\left(K_{1}, \ldots, K_{n}\right)
\]
\item $V$ is positive, and in each argument,$V$ is monotone.	 
 
  \end{enumerate}
 
 
 
 
 \subsection{The Brunn-Minkowski Theorem}
 
 Brunn-Minkowski Theorem  basically says that the function 
 \[s  \mapsto \sqrt[n]{V(s K+(1- s) L)}, \quad s \in[0,1]
\]
 is concave for $ K, L \in \mathcal{K}^n $.
We will get inequalities for mixed volumes, and as a consequence also get the famous isoperimetric inequality from concavity of above function
\\\\
We will be using result of this following lemma further for proving Brunn-Minkowski Theorem
\begin{lemma} 
\label{l:1}
For $a \in(0,1)$ and $r, s, t>0$
\[
\left(\frac{a}{r}+\frac{1-a}{s}\right)\left[a r^{t}+(1-a) s^{t}\right]^{\frac{1}{t}} \geq 1
\]
with equality, iff $r=s$
\end{lemma} 
 \begin{proof}
 
 For proving ths result we will be using the fact that the function $x \rightarrow lnx $ is strictly concave \\
 so taking ln of the expression as argument of expression is always positive. we get 
 $$\ln \left\{\left(\frac{a}{r}+\frac{1-a}{s}\right)\left[a r^{t}+(1-a) s^{t}\right]^{\frac{1}{t}}\right\}$$
 
$$ =\frac{1}{t} \ln \left(a r^{t}+(1-a) s^{t}\right)+\ln \left(\frac{a}{r}+\frac{1-a}{s}\right)$$
Now using strict concavity of  function  $x \rightarrow lnx $ we get 
$$\frac{1}{t} \ln \left(a r^{t}+(1-a) s^{t}\right)+\ln \left(\frac{a}{r}+\frac{1-a}{s}\right)
  \geq \frac{1}{t}\left(a \ln r^{t}+(1-a) \ln s^{t}\right)+a \ln \frac{1}{r}+(1-a) \ln \frac{1}{s}$$
  
  
$$ =  \left(a \ln r+(1-a) \ln s\right)+a \ln \frac{1}{r}+(1-a) \ln \frac{1}{s}$$
$$\quad = 0$$
 
 Now as logarithm is strict monotone function we get required result 
 \[
\left(\frac{a}{r}+\frac{1-a}{s}\right)\left[a r^{t}+(1-a) s^{t}\right]^{\frac{1}{t}} \geq 1
\]
 clearly equality follows when $r = s $
 \end{proof}
 
 
 
 
 
 
 
 
 
 \begin{theorem}(\textbf{Brunn-Minkowski Theorem})
 \label{thm:20}
For convex bodies $J, K \in \mathcal{K}^{n}$ and $a \in(0,1)$
\begin{equation}
\label{eq54}
\sqrt[n]{V(a J+(1-a) K)} \geq a \sqrt[n]{V(J)}+(1-a) \sqrt[n]{V(K)}
\end{equation} \label{eq:4.19}
with equality, iff $J$ and $K$ lie in parallel hyperplanes or $J$ and $K$ are homothetic( see \ref{def:2.1})
  \end{theorem}
\begin{proof}
Based on the dimension of the convex body, we consider 4 cases.
 \hfill \break
 
\textbf{Case 1} $: \quad \text{When } J$ and $K$ lie in parallel hyperplanes. Then  $a J+(1-) K$  also lies in a hyperplane,\quad so that $$V(J)=V(K)=0$$    $$\text{and also}$$ $$V(a J+(1-a) K)=0$$
 hence done for this case.
 \hfill \break
\textbf{Case 2}: \quad  When $\operatorname{dim} J \leq n-1$ and $\operatorname{dim} K \leq n-1,$ but $J$ and $K$ do not lie in parallel hyperplanes, for all $a \in(0,1)$, $\operatorname{dim}(J+K)=n .$ and hence  $\operatorname{dim}(a J+(1-a) K)=n \text{ also }.$ 
\hfill \break
$V(J) = V(K)= 0$ as $\operatorname{dim} J,K \leq n-1 $ Therefore for all $a \in(0,1)$
 $$
a \sqrt[n]{V(J)}+(1-a) \sqrt[n]{V(K)}=0 \leq \sqrt[n]{V(a J+(1-a) K)}
$$
\hfill \break
\textbf{Case 3}: \quad  When $\operatorname{dim } J \leq n-1$ and $\operatorname{dim} K=n$ or vice versa
\hfill \break 
Then, for $x \in J,$ we get
\[
a x+(1-a) K \subset a K+(1-a) K
\]
and Consequently
\[
(1-a)^{n} V(K)=V(a x+(1-a) K) 
\]
as adding a point $a x $ is translation of $(1-a) K$ and won't effect Volume. 
and
$$ V(a x+(1-a) K) \leq V(a 
J+(1-a) K)$$
and equality occur, iff $J=\{x\}$ 
\hfill \break
\textbf{Case 4}:When $\operatorname{dim} J=\operatorname{dim} K=n .$  We are allowed to take $V(J)=V(K)=1 .$ as for general $J, K,$  we  can take
\[
  \widetilde{J}:=\frac{1}{\sqrt[n]{V(J)}} J, \quad \widetilde{K}:=\frac{1}{\sqrt[n]{V(K)}} K
\]
and
\[
\widetilde{a}:=\frac{a \sqrt[n]{V(J)}}{a \sqrt[n]{V(J)}+(1-a) \sqrt[n]{V(K)}}
\]
Then we have
\[
    \sqrt[n]{V({\widetildea \widetilde{ J} }+(1-\widetilde{a}) \widetilde{K})} \geq 1
\]
Now  as $V(\widetilde{J}) = V(\widetilde{K})$ = 1 so $$\sqrt[n]{V(J)}+(1-a) \sqrt[n]{V(K)}=a+(1-a)=1$$
hence $\widetilde{J}$ and $\widetilde{K}$
follow the BRUNN-MINKOWSKI Theorem,  and also, $J$ and $K$ are homothetic, iff $\widetilde{J}$ and $\widetilde{K}$ are homothetic.Therefore,Now  we have to just prove that given condition that $V(J)=V(K)=1$, We have 
\[
V(a J+(1-a) K) \geq 1
\]
with equality iff $J, K$ are translates of each other.
\\
 We define centre of gravity for an $n$ -dimensional convex body $S$ to be the point $c \in \mathbb{R}^{n}$ such that
\[
\langle c, u\rangle=\frac{1}{V(M)} \int_{S}\langle x, u\rangle d x
\]
for all $u \in S^{n-1} .$ \newpage
By translating  $J$ and $K$ we can have their centre of gravity at 0 as the Volume is translation invariant,the equality case then just reduces to claim hat $K=L$
\hfill \break
Now by induction on $n$, We prove  BRUNN-MINKOWSKI Theorem 
\hfill \break
For $n=1$, 1 - dimensional volume is linear so Brunn-Minkowski inequality easily follows 
\hfill \break 
Along with above conclusion we also get that equality corresponds to the fact that two convex bodies (which are compact intervals in $\mathbb{R}$), in $\mathbb{R}$  are homothetic.
\hfill \break 
Now for $n \geq 2$ if the Brunn-Minkowski theorem is true in dimension $n-1 $
\\
we choose a unit vector $u \in S^{n-1}$ and denote 
the hyperplane $E_{\lambda}$ in the direction $u$ with  distance(signed) $\lambda \in \mathbb{R} $ from the origin defined by 
\[
E_{\lambda}:=\{x:\langle\ x, u\rangle=\lambda\}, 
\]
Now since by Fubini's theorem  
\[
V(K \cap\{x: \langlex x, u\rangle \leq \beta\})=\int_{-h_{ K(-u)}}^{\beta} v\left(K \cap E_{\lambda}\right) d \lambda
\] 
The function
\[
g:\left[-h_{K}(-u), h_{K}(u)\right] \rightarrow[0,1], \quad \gamma \mapsto V(K \cap\{x:\langle x, u\rangle \leq  \gamma \}
\]
would be  continuous and strictly increasing\\
where $-h_{K}(-u), h_{K}(u)$ are support function of K at $u$ and $-u$ and are extreme upper point and extreme lower point respectively.
\hfill \break
  Map $\lambda \mapsto v\left(K \cap E_{\lambda}\right)$ is continuous on $\left(-h_{K}(-u), h_{K}(u)\right),$ where $v$ is area functional.
  %some doubt in v is  area measure 
   $g$ is differentiable function on $\left(-h_{K}(-u), h_{K}(u)\right)$ and $$g^{\prime}(\beta)=v\left(K \cap E_{\beta}\right)$$
  \\
  since $g$ is invertible, inverse function of $g$, $ \gamma:[0,1] \rightarrow\left[-h_{K}(-u), h_{K}(u)\right],$ which is also a continuous function and strictly increasing  function satisfies $ \gamma(0)=-h_{K}(-u),  \gamma(1)=h_{K}(u)$ and
\[
 \gamma^{\prime}(\tau)=\frac{1}{g^{\prime}( \gamma(\tau))}=\frac{1}{v\left(K \cap E_{ \gamma(\tau)}\right)}, \quad  \text{where } \tau \in(0,1)
\]
Similarly, For K
The function
\[
f:\left[-h_{K}(-u), h_{K}(u)\right] \rightarrow[0,1], \quad \delta \mapsto V(K \cap\{x:\langle x, u\rangle \leq  \delta \}
\]
and in the same way
we get its inverse function $ \delta:[0,1] \rightarrow\left[-h_{L}(-u), h_{L}(u)\right]$     with 
\[
\delta^{\prime}(\tau)=\frac{1}{v\left(L \cap E_{\delta(\tau)}\right)}, \quad  \text{where } \tau \in(0,1)
\]
Because of
\[
a\left(K \cap E_{g(\tau)}\right)+(1-a)\left(L \cap E_{\delta(\tau)}\right) \subset(a K+(1-a) L) \cap E_{a \gamma(\tau)+(1-a) \delta(\tau)}
\]
for $a, \tau \in[0,1],$ we obtain from the inductive assumption
\[
\begin{array}{l}
V(a J+(1-a) K) \\\\
\quad=\int_{-\infty}^{\infty} v\left((a J+(1-a) K) \cap E_{\lambda}\right) d \lambda \\\\
\quad=\int_{0}^{1} v\left((a J+(1-a) K) \cap E_{a \gamma(\tau)+(1-a) \delta(\tau)}\right) \\\\
 \quad  \quad \quad \quad \quad \quad \quad \quad \quad \times
\left(a \gamma^{\prime}(\tau)+(1-a) (\delta)^{\prime}(\tau)\right) d \tau
\\\\\\
\geq \int_{0}^{1} v\left(a\left(J \cap E_{\gamma(\tau)}\right)+(1-a)\left(K \cap E_{\delta(\tau)}\right)\right)
\\\\
 \quad \quad \quad \quad \quad \quad \quad \quad \quad \quad \quad
 
 \times
 
 \left\{\frac{a}{v\left(J \cap E_{\gamma(\tau)}\right)}+\frac{1-a}{v\left(K \cap E_{\delta(\tau)}\right)}\right\} d \tau \\\\\\
\geq \int_{0}^{1} \left\{  a^{n-1} \sqrt{v\left(K \cap E_{\gamma(\tau)}\right)}+(1-a) \sqrt[n-1]{v\left(L \cap E_{\delta(\tau)}\right)} \right\}^{n-1} \\\\
\quad \quad \quad \quad \quad \quad \quad \quad \quad \quad \quad \times \left\{ \frac{a}{v\left(J \cap E_{\gamma(\tau)}\right)}+\frac{1-a}{v\left(K \cap E_{\delta(\tau)}\right)}\right\} d \tau
\end{array}
\]
Choosing $r:=v\left(J \cap E_{\gamma(\tau)}\right), s:=v\left(K \cap E_{\delta(\tau)}\right)$ and $t:=\frac{1}{n-1},$
\\
applying  Lemma \ref{l:2} on above expression written inside integration is $\geq 1,$ and on the limi of 0 and 1 whole integrand also get  $\geq 1,$ which gives the required result.
\\
 Now For equality case assume that equality occur so that
\[
V(a J+(1-a) K) = 1 
\]
 For above expression equal to $1,$ in our last estimation we must have equality, which means that the integrand  is equal to $1,$ for all $\tau .
 $ and this the equality case(iff $r = s$) of same Lemma \ref{l:2}
 \\
 So we have 
\[
v\left(J \cap                E_{\gamma(\tau)}\right)=v\left(K \cap E_{\delta(\tau)}\right), \quad \text { for all } \tau \in[0,1]
\]
Hence $g^{\prime}= \delta^{\prime},$ so $\gamma- \delta$ is a constant.\\
Because the centre of gravity of $J$ is at the origin, we get 
\\
\[
0=\int_{J}\langle x, u\rangle d x=\int_{\gamma(0)}^{\gamma(1)} \lambda v\left(K \cap E_{\lambda}\right) d \lambda=\int_{\gamma(0)}^{\gamma(1)} \lambda \gamma^{\prime}(\lambda) d \lambda=\int_{0}^{1} \gamma(\tau) d \tau
\]
where the change of variables $\lambda=\gamma(\tau)$ was used. In an analogous way,
\[
0=\int_{0}^{1} \delta(\tau) d \tau
\]
Consequently,
\[
\int_{0}^{1}(\gamma(\tau)-\delta(\tau)) d \tau=0
\]
and therefore $\gamma=\delta .$ In particular, we obtain
\[
h_{J}(u)=\delta(1)=\delta(1)=h_{K}(u)
\]
since $u$ was arbitrary, $V(a J+(1-a) K)=1$ implies $h_{J}=h_{K},$ and hence $J=K$ \\
Conversely,when $J=K$ 
$$V(a J+(1-a) K)=V(a J+(1-a) J) =  V(J) = 1 $$
Hence Proved
 \end{proof}
 %change to modify language please looks like totally book
 \hfill \break
 
 \begin{corollary}
 
  function $ S(b):b\rightarrow \sqrt[n]{V(b J+(1-b) K)}$ } is concave on $[0,1] .$ 
  \end{corollary}
  \begin{proof}
      
 \[
S(b):=\sqrt[n]{V(b J+(1-b) K)}
\] 
is  easily using above theorem \ref{thm:20}
\hfill \break
we have to prove that 
 $$ S(a x+(1-a) y) \geq a S(x)+(1-a) S(y) $$
 
Let $x, y, \text{a} \in[0,1],$ 
\hfill \break
$S(ax+(1-a)y) &=\sqrt[n]{V([ax+(1-a)y] J+[1-a x-(1-a) y] K)}
\hfill \break
\quad \quad \quad \quad \quad \quad \quad \quad 
 =\sqrt[n]{V(a[x J+(1-x) K]+(1-a)[y J+(1-y) K])} 
\hfill \break
 \quad \quad \quad \quad \quad \quad \quad \quad   
 \geq a \sqrt[n]{V(x J+(1-x) K)}+(1-a) \sqrt[n]{V(y J+(1-y) K)}
    \quad \quad \quad \quad \quad \text{from Theorem } \ref{thm:20}
\hfill \break
 \quad \quad \quad \quad \quad  \quad  \quad \quad = a S(x)+(1-a) S(y)
$
  \end{proof}
       
 
 
 
 
 
 
 
 
 
 
 
 
 
\begin{theorem}
 
For $K, L \in \mathcal{K}^{n}$
\[
V(K, \ldots, K, L)^{n} \geq V(K)^{n-1} V(L)
\]
with equality, if and only if $\operatorname{dim} K \leq n-2$ or $K$ and $L$ lie in parallel hyperplanes or $K$ and $L$ are homothetic.
\end{theorem} 
\begin{proof}
    
 For $\operatorname{dim} K \leq n-1$, the inequality holds since the right-hand side is zero. Moreover, we then have equality, if and only if either $\operatorname{dim} K \leq n-2$ or $K$ and $L$ lie in parallel hyperplanes (compare Exercise 3.3 .1 ). Hence, we now assume $\operatorname{dim} K=n$ By Theorem 3.4 .2 (similarly to the preceding remark), it follows that the function
\[
f(t):=V(K+t L)^{\frac{1}{n}}, \quad t \in[0,1]
\]
is concave. Therefore
\[
f^{+}(0) \geq f(1)-f(0)=V(K+L)^{\frac{1}{n}}-V(K)^{\frac{1}{n}}
\]
since
\[
f^{+}(0)=\frac{1}{n} V(K)^{\frac{1}{n}-1} \cdot n V(K, \ldots, K, L)
\]
we arrive at
\[
V(K)^{\frac{1}{n}-1} \cdot n V(K, \ldots, K, L) \geq V(K+L)^{\frac{1}{n}}-V(K)^{\frac{1}{n}} \geq V(L)^{\frac{1}{n}}
\]
where we used the Brunn-Minkowski inequality in the end (with $t=\frac{1}{2}$ ). This implies the assertion. Equality holds if and only if equality holds in the Brunn-Minkowski inequality, which yields that $K$ and $L$ are homothetic.      
 \end{proof}     
 
 
 
 
 
 \begin{corollary}\textbf{(Isoperimetric Inequality)}       
 
 
 Let $K \in \mathcal{K}^{n}$ be a convex body of dimension $n .$ Then,
\[
\left(\frac{F(K)}{F(B(1))}\right)^{n} \geq\left(\frac{V(K)}{V(B(1))}\right)^{n-1}
\]
Equality holds, if and only if $K$ is a ball.
 \end{corollary}
 \begin{proof}
     
Proof. We put $L:=B(1)$ in Theorem 3.4 .3 and get
\[
V(K, \ldots, K, B(1))^{n} \geq V(K)^{n-1} V(B(1))
\]
or, equivalently,
\[
\frac{n^{n} V(K, \ldots, K, B(1))^{n}}{n^{n} V(B(1), \ldots, B(1), B(1))^{n}} \geq \frac{V(K)^{n-1}}{V(B(1))^{n-1}}
\]
3.4. THE BRUNN-MINKOWSKI THEOREM
The isoperimetric inequality states that, among all convex bodies of given volume (given surface area $),$ the balls have the smallest surface area (the largest volume). Using $V(B(1))=\kappa_{n}$ and $F(B(1))=n \kappa_{n},$ we can re-write the inequality in the form
\[
V(K)^{n-1} \leq \frac{1}{n^{n} \kappa_{n}} F(K)^{n}
\]
For $n=2$ and using the common terminology $A(K)$ for the area (the "volume" in $\mathbb{R}^{2}$ ) and $L(K)$ for the boundary length (the "surface area" in $\mathbb{R}^{2}$ ), we obtain
\[
A(K) \leq \frac{1}{4 \pi} L(K)^{2}
\]
and, for $n=3$
\[
V(K)^{2} \leq \frac{1}{36 \pi} F(K)^{3}
\]
Exchanging $K$ and $B(1)$ in the proof above yields a similar inequality for the mixed volume $V(B(1), \ldots B(1), K),$ hence we obtain the following corollary for the mean width $\bar{B}(K)$
  \end{proof}
 
 
    
    
    
    
    
    
    
    
    
    
    
    
    
    
    
    
    
    
    
    
    
    
    
    
    
    
    
    
    
    
    
    
    
    
    
    
    
    
    
    
    
    
    
    
    
    
    
    
    
    
    
    
    
    
    
    
    
    
    
    
    
    
    
    
    
    
    
    
    
    
    
    
    
    
    
    
    
    
    
    
    
    
    
    
    
    
    
    
    
    
    
    
    
    
    
    
    
    
    
    
    
    
    
    
    
    
    
    
    
    
    
    
    
    
    
    
    
    
    
    
    
    
    
    
    
    
    
    
    
    
    
    
    
    
    
    
    
    
    
    
    
    
    
    
    
    
    
    
    
    
    
    
    
    
    
    
    
    
    
    
    
    
    
    
    
    
    
    
    
    
    
    
    
    \chapter{Isoperimetric Inequaity for Graph}
    \section{Introduction}
  
 
  Isoperimetric Inequality for a Graph G is defined in similar way as defined in $\mathhbb{R}^n$. it involves finding a subset of the Vertex set V(G) of G among all subsets of fixed size, which has the smallest edge-boundary. These subsets are called Isoperimetric Sets. Further  Isoperimetric Sets are given for some trivial and non-trivial classes of Graph. \par 
    from paper [1], Isoperimetric number $i(G)$ and $i_k{G}$ are defined, and some basic properties of $i(G)$  are discussed. Then the main theorem for this project is proved from section 6 which uses theorem from section 4 regarding lower bound for $i_k{G}$ and for i(G) in terms of the second eigenvalue of difference Laplacian matrix D and uses theorem regarding bounding expectation of the second eigenvalue of random 2k-regular Graph from different paper [2]. For proof of this theorem from [2], some results are used without proof from paper [3] and [4].\par
  
\susection{Isoperimetric Inequality for Graph G}
For a graph G with vertex set V and edge set E, Isoperimetric Inequality is to find a function $\mathcal{F}$, called best  Isoperimetric function such that  \begin{center}
  $|\partial \Omega| \geq \mathcal{F}(|\Omega|)$  
\end{center}   
for every non empty subset $\Omega$ of vertices set V, where $\partial \Omega$ is the boundary of $\Omega$  and defined as the set of edges of the graph connecting vertices of $\Omega$ with vertices of its complement $\bar{\Omega} = V \backslash \Omega $ and $\mathcal{F}$ is defined as  \begin{center}
    $\mathscr{F}(k)=\min _{|\Omega|=k}|\partial \Omega|$\par
\end{center} 
 Best Isoperimetric function or corresponding Sharp Isoperimetric Inequality is known only for few classes of graph e.g.
from combinatorial arguments 
\par
\begin{itemize}
    \item For The complete graph $K_{n}$ inequality is $|\partial \Omega|=(n-|\Omega|) | \Omega|$ 
    \item For The cycle $C_{n}$ it is  $|\partial \Omega|\geq 2$ for $|\Omega| \neq n$.
    \item For The infinite d-regular tree it is $|\partial \Omega| \geq(d-2)| \Omega|+2$\par
\end{itemize}
 
  
     
     Other Non-trivial classes for which it is known are some families of cartesian products of graphs. like 
n-cube $Q_{n}$, 
grid $[k]^{n}$,
and lattice $\mathbb{Z}^{n}$.
\par
 Apart from these combinatorial techniques There are eigenvalue techniques which give good Isoperimetric Inequality in general \\\\
 
 Using eigenvalue techniques, 
 Isoperimetric Inequality for d-regular Graph in terms of second eigenvalue $\lambda_2$\ is following. \par
  
 Let $\lambda_{2}$ be the second smallest eigenvalue of Laplacian L of G then for a random d-regular graph $|\partial \Omega| \geq \lambda_{2} \frac{|\Omega||\bar{\Omega}|}{|V|}$  is a good approximation of best Isoperimetric function from the work of Friedman and Bollobas  \par 
     
     
     \subsection{Laplacian matrix L of G }
     
      Laplacian matrix L of G is defined as  $L = D - A$ where  D be n x n diagonal matrix $D_{jj} = d_{j}$ and A is adjacency matrix of the Graph G.\par
      
          
            
          It is clear that   $L_{ij} = -1$ for $i\neq j$  for $ij\in E(G)$,  $L_{ii}$ = $d_i$(degree of ith vertex) and   $L_{ij} = 0$ otherwise.\par
          
          It can be easily seen that L is asymmetric and a positive definite  matrix so its eigenvalues are real and positive. \\
         
          Let $\mu_{1} \leq \mu_{2} \leq \cdots \leq \mu_{n}$, $\mu_{i} \geq 0$  be  eigenvalues of L then smallest eigen value $\mu_{1} = 0$ as row sum of L  be 0 and all-ones vector(each entry is 1) is an eigen vector of eigen value 0. That's why we consider second smallest eigen value of L. Largest eigen value of Adjacency Matrix is found to be d for regular-d Graph. \par
          
          
            
         
In general, there is no simple relationship between the eigenvalues of
A and the eigenvalues of L but In case of d-regular graphs, d is the
common vertex degree and $L = dI - A$  so if $\lambda_{1} \geq \lambda_{2}....\geq \lambda_{n}$  be the eigenvalues of A, we have eigenvalues of L are $\mu_{j} = d - \lambda_{j}$ (each eigenvalue of sum of two matrix is sum of corresponding  individual eigenvalue if both matrix commute multiplicatively). \\\\
Throughout the report, We will denote $\lambda_{2} $ to be a second largest eigenvalue of the adjacency matrix A and $\mu_{2}$ to be the second smallest eigenvalue of corresponding Laplacian matrix of A for 2k-regular Graph.  
          
           
         
           \subsection{Isoperimetric Number of Graph G}
           
           
           The Isoperimetric number of Graph G is defined as \par
           \begin{center}
               $i(G)=\min _{\Omega} \frac{|\partial \Omega|}{|\Omega|}$
           \end{center}
           where minimum is taken over all nonempty subsets $\Omega$ of V satisfying\par $|X| \leq \frac{1}{2}|V(G)|$.
\\           We can think of the quantity  $\frac{|\partial \Omega|}{|\Omega|}$ as the average boundary degree of X. \par
           i(G) can also be defined in following way \par
           \begin{center}
               $i(G)=\min \frac{|E(X, Y)|}{\min \{|X|,|Y|\}}$
           \end{center}
            where minimum runs over all partitions of V into nonempty subsets X and Y such that  $V=X \cup Y$ and $E(X, Y)=\partial X=\partial Y$ are the edges between X and Y.
           \par
           The isoperimetric constant can be understood as a measure of how easy
it is to disconnect a large part of the graph. \par
           For getting i(G) we have to find small edge-cut $E(X, Y)$ separating as large as possible  subset X from the remaining larger part Y assuming $|X|\leq |Y|$ so in this way i(G) can serve as measure of connectivity of Graph. \\\\
           Some results for different classes of Graphs are following \par
           \begin{enumerate}
             
          
               \item  $i(G)=0$ if and only if $G$ is disconnected \par
               Proof. we can choose  any part of the graph which is not connected to rest of Graph and will get $i(G)=0$ \par
               \item  If $G$ is $k$ -edge-connected(require at least k edge to break connectivity) then $i(G) \geq 2 k /|V(G)|$ \par
        Proof. For any chosen subset X of V,we have  $|\partial X| \geq k$ and we can take $|X|$ to be $|V(G)|/2$ and hence done\par
        \item  If $\delta$ is the minimal degree of vertices in $G$ then $i(G) \leq\delta$ \par
        \item  If  & $e=u v$ is   an  edge  of G  and $|V(G)| \geq 4$ \par then  $i(G) \leq [deg(u)+deg(v)-2] / 2$ \par
            \end{enumerate}
          
           
          
         
        Isoperimetric number for some classes of Graph are following \par
       \begin{enumerate}
            \item   For the complete graph $K_{n}, i\left(K_{n}\right)=\lceil n / 2\rceil$ \par
            \item The cycle $C_{n}$ has $i\left(C_{n}\right)=2 /\lfloor n / 2\rfloor$ \par
            \item  The path $P_{n}$ on $n$ vertices has $i\left(P_{n}\right)=1 /\lfloor n / 2\rfloor$
        \end{enumerate}
        \section{Bounding $\mathrm{E}\left|\lambda_{2}(G)\right|$ of random 2d-regular Graph}
        Here  we will be con
        
        
     \begin{theorem} 
     $ \left(\mathbf{A} \right) $ For $G \in{G}_{n, 2d}$ we have
$$
\mathbf{E}\left|\lambda_{2}(G)\right| \leq 2 \sqrt{2 d-1}\left(1+\frac{\log d}{\sqrt{2 d}}+o\left(\frac{1}{\sqrt{d}}\right)\right)+O\left(\frac{d^{3 / 2} \log \log n}{\log n}\right)
$$  and we have \begin{center}
     $\mathrm{E|\lambda_{2}(G)|}$ \leq $O(d^{1/2})$ when d is fixed and $n\rightarrow \infty
     $
\end{center}
 \\\\
  where E denotes the expected
value over ${G}_{n, 2d}$      and more generally we have\\\\ $\mathbf{E}\left|\lambda_{2}(G)\right|^{m} \leq\left(2 \sqrt{2 d-1}\left(1+\frac{\log d}{\sqrt{2 d}}+O\left(\frac{1}{\sqrt{d}}\right)\right)+O\left(\frac{d^{3 / 2} \log \log n}{\log n}\right)\right)^{m}$ \par
for any $ m \leq 2\lfloor\log n\lfloor\sqrt{2 d-1} / 2\rfloor / \log d\rfloor$\\\\
$ \left(\mathbf{B}\right) $   As a corollary, For a $\beta\geq 1$ we have \par
$\left|\lambda_{2}(G)\right| \geq\left(2 \sqrt{2 d-1}\left(1+\frac{\log d}{\sqrt{2 d}}+O\left(\frac{1}{\sqrt{d}}\right)\right)+O\left(\frac{d^{3 / 2} \log \log n}{\log n}\right)\right) \beta$\\\\
 with probability
$$
\leq \frac{\beta^{2}}{n^{2}\lfloor\sqrt{2 d-1} / 2\rfloor \log \beta / \log d}
$$   \\\\
 \end{theorem}
\begin{proof}
 The standard approach to estimating eigenvalues is to estimate the trace of high power of the adjacency matrix.\par
First we define ${G}_{n, 2d}$, probability space of 2d-regular Graphs on n vertices    we choose d permutation of the numbers ${1,\ldots n}$ each permutation equally likely and construct a direct Graph $G = (V,E) $ with  vertex set $V=\{1, \ldots, n\}$ and edges\par
$E=\left\{\left(i, \pi_{j}(i)\right),\left(i, \pi_{j}^{-1}(i)\right) | j=1, \ldots, d \quad i=1, \ldots, n\right\}$ \\\\
Constructed graph G will have possibly multiple edges and self-loops, and which is symmetric in the sense that for every edge in the Graph, the opposite edge is also there. Although G is directed, we can view it as an undirected graph by
replacing each pair of edges $(i, \pi(i)), (\pi(i), i) $ with one undirected edge.\par In this way, Each vertex of this Graph has its degree 2d, and  We denote
this probability space of random graphs by $G_{n,2d}$. \\\\
Let $A$ be the adjacency matrix of a graph $G \in \mathscr{G}_{n, 2 d}$. Let $\Pi$ be the alphabet of symbols
$$
\Pi=\left\{\pi_{1}, \pi_{1}^{-1}, \pi_{2}, \ldots, \pi_{d}^{-1}\right\}
$$
where  $\pi_{1}, \ldots, \pi_{d}$ are  $d$ permutations from which $G$ was constructed, and any word, $w=\sigma_{1} \ldots \sigma_{k}$ of $\Pi^{*}$ as the permutation which is the composition of the permutations $\sigma_{1}, \ldots, \sigma_{k} .$ \par
Define  \begin{center}
   $i \stackrel{w}{\rightarrow} j \equiv\left\{\begin{array}{ll}
1 & \text { if  }  w(i)=j \\
0 & \text { otherwise }
\end{array}\right\}$ 
\end{center}
Then  $i, j$ -th entry of $A^{k}$ will be 
$$
\sum_{w \in \Pi^{k}} i \stackrel{w}{\rightarrow} j
$$
What we want is  to estimate expectation of  trace, so we need to take the expectation of the above sum only for  $i = j$      
 In evaluating $i \stackrel{w}{\rightarrow} i,$  $\pi \pi^{-1}$ with $\pi \in \Pi $    will fix the vertex so we can cancel this type of pairs in  $w$.\par
 We define a word $w \in \Pi^{k}$ to be irreducible if $w$ has no pair of consecutive letters of the form $\pi \pi^{-1}$ and denote the set of irreducible words of length $k$ by Irred $_{k} .$\par
 For 1st position of a word, first letter has  2d choices and then its next letter can not be its inverse so all other k - 1 letters will have only 2d - 1 choices so
  Irred $_{k}$ has its  size $2 d(2 d-1)^{k-1} .$ \par 
  It turns out that to estimate the second eigenvalue it suffices
to get an estimate of the form
\begin{center}
   $\sum_{\boldsymbol{w} \in \text { Irred }_{\boldsymbol{k}}} i \stackrel{w}{\rightarrow} i=2 d(2 d-1)^{k-1} \frac{1}{n}+$ error 
\end{center}
for all fixed i with some small error term which will be bounded later.\par
 1/n comes because we expect  that words in $Irred_{k} $ send a fixed vertex to each  others with more or less equal probability which is 1/n . \\\\
We have following result as a corollary and will be using this result to estimate the expected sum of the k-th powers of the eigenvalues \par
 \text { For any fixed } i, $k \geq 1$, \text { and } $d-2>\sqrt{2 d-1} / 2$ \text { (i.e. } $d \geq 4)$ we have
$$
\mathrm{E}\left\{\sum_{w \in \mathrm{Irred}_{k}} i \stackrel{w}{\rightarrow} i\right\}=2 d(2 d-1)^{k-1}\left(\frac{1}{n}+\operatorname{error}_{n, k}\right)
$$
where
$$
\operatorname{error}_{n, k} \leq(c k d)^{c}\left(\frac{k^{2 \sqrt{2 d}}}{\left.n^{1+\lfloor\sqrt{2 d-1} / 2}\right\rfloor}+\frac{(2 d-1)^{-k / 2}}{n}\right)
$$
All words in $\Pi^{k}$ can be reduced to irreducible words by removing occurrences of $\pi \pi^{-1}$ in the word, and this irreducible word is independent of how the reducing was done. \par 
Let  $p_{k,s}$ is the probability that a random word in  $\Pi^{k}$ reduces to an irreducible word of size s then breaking sum over words in  $\Pi_k$ to sum over irreducible words of size s  we get  
$$
\frac{1}{(2 d)^{k}} \mathbf{E}\left\{\sum_{w \in \Pi^{k}} i\rightarrow i\right\}=p_{k, 0}+\sum_{s=1}^{k} p_{k, s} \frac{1}{2 d(2 d-1)^{s-1}} \mathrm{E}\left\{\sum_{w \in \mathrm{Irred}_{s}} i \stackrel{w}{\rightarrow} i\right\}
$$
  since $\sum_{s} p_{k, s}=1,$ we have
$$ 
\frac{1}{(2 d)^{k}} \sum_{i} \mathrm{E}\left\{\sum_{w \in \Pi^{k}} i \stackrel{w}{\rightarrow} i\right\}=1+(n-1) p_{k, 0}+\sum_{s=1}^{k} n p_{k, s} \operatorname{error}_{n, s} 
$$    ..........(1)\par
We are using following result for $p_{2k},{2s}$ without proof( proof is in paper [3])  \par
$$
p_{2 k, 2 s} \leq \frac{2 s+1}{2 k+1}\left(\begin{array}{c}
2 k+1 \\
k-s
\end{array}\right)\left(\frac{1}{2 d}\right)^{k-s}\left(1-\frac{1}{2 d}\right)^{2 s-1}
$$
 Note that k and s can not have different parity as  $\pi$ and $\pi^{-1}$ will get vanish in pair\par
   from proof of above from [3] it is seen that(Using here as a result)  
$$ p_{2 k, 0} \geq \frac{1}{2 k+1}\left(\begin{array}{c}
2 k+1 \\
k
\end{array}\right) \frac{(2 d-1)^{k}}{(2 d)^{2 k}}$$
It follows that for any graph of degree $2 d$,
$$
\sum_{i=1}^{n} \lambda_{i}^{2 k} \geq(2 d)^{2 k}(n-1) p_{k, 0} \approx(n-1) 2^{2 k}(2 d-1)^{k}
$$
so that taking $2 k$ slightly less than $2 \log _{d} n$ results 
$$
\left|\lambda_{2}\right| \geq 2 \sqrt{2 d-1}+O\left(\frac{1}{\log _{d} n}\right)
$$
Now we take $k=2$ $\lfloor\log n\lfloor\sqrt{2 d-1} / 2\rfloor / \log d\rfloor,$ so that $k$ is even, and calculate
using the simplified bound
$$
p_{2 k, 2 s} \leq 2^{2 k}\left(\frac{1}{2 d}\right)^{k-s}
$$
It is easy to see that the dominant terms of the summation over $s$ in equation (1) are $s=1$ and $s=k,$ and therefore
$$
\mathrm{E}\left\{\sum_{i=2}^{n} \lambda_{i}^{k}\right\} \leq n^{1+\frac{\log 2}{\log \alpha}(c k d)^{c} k^{2 \sqrt{2 d}}(2 \sqrt{2 d} \sqrt{\frac{2 d}{2 d-1}})^{k}}
$$
Taking $k$ -th roots, applying Hölder's  inequality, and noticing that
$$
\left(n^{1+\frac{\log 2}{\log d}}\right)^{1 / k}=1+\frac{\log d}{\sqrt{2} \bar{d}}+O\left(\frac{1}{\sqrt{d}}\right)
$$
and 
$$
(c k d)^{c / k} k^{2 \sqrt{2 d} / k}=1+O\left(\frac{\log d \log \log n}{\log n}\right)
$$
and that
$$
k \leq \frac{1}{c d^{3 / 2}} n^{\frac{1}{c \sqrt{a}}}
$$
for
$$
\frac{\log n}{\log \log n} \geq c^{\prime} \sqrt{d}
$$
proves Theorem(A)  \par 
From here it is clear that $\mathrm{E|\lambda_{2}(G)|}$ \leq $O(d^{1/2})$ as $\frac{ \log \log n}{\log n}\rightarrow 0$ in limit  as n goes to  infinity 
\end{proof} 
\begin{theorem} 
 Let $G$ be a graph on $n$ vertices and let $\mu_{2}$ be the second smallest eigenvalue of its difference Laplacian matrix D. Then for every k, 
$1 \leq k \leq n-1$
$$
i_{k}(G) \geq \frac{(n-k) \mu_{2}}{n}
$$
and, consequently, $i(G) \geq \mu_{2} / 2$ \par
\end{theorem} 
\begin{proof}
for the second smallest eigenvalue $\mu_{2}$ of $D$ and for an arbitrary $X \subseteq V(G)$ the following relation holds(from paper [4]):
$$
\mu_{2} \leq|\partial X|\left(\frac{1}{|X|}+\frac{1}{|V \backslash X|}\right)
$$
putting $|X| = k $ and $|V\backlash X| = n-k $ and using definition of $i_k(G)$ 
 $ \mu_{2} \leq|\partial X|\left(\frac{1}{k}+\frac{1}{n-k}\right)$ = $|\partial X|\left(\frac{n}{k(n-k)}\right) $ \par
 which gives required result\\\\
 \end{proof} 
 
 \begin{mydef}
 
 
$F(n, k)$:= max$\left\{i(G)| \textit{G is k-regular  n vertices} \right\}$ 
and
$$
f(k):=\lim _{n \rightarrow \infty} \sup F(n, k)
$$
\end{mydef}
 
 
\begin{theorem}
 
 
  \begin{center}
    $f(2 k) \geq k-O\left(k^{1 / 2}\right)$ 
     \end{center}
    \end{theorem}
   
   
\begin{proof} From theorem 5.1, we have  $\mathrm{E|\lambda_{2}(G)|}
    \leq O(k^{1/2})$ when d is fixed and $n\rightarrow \infty$ for second largest eigen value $\lambda_{2}$ of Adjacency Matrix A, so second smallest eigen value($\mu_{2}$ of corresponding laplacian Matrix (or second eigen value of L) will be $2k - \lambda_{2}$ \\
     
     
   As $f(k)$ is limsup of $F(n,k)$s and each $F(n,k)$ is maximum over $i(G)$ of k-regular graphs G
   so $i(G)\leq f(2k)$
   
    From Theorem 5.2 we have
    \begin{center}
        $i(G) \geq \frac{\mu_2}{2} = \frac{2k - \lambda_{2}}{2} $ 
    \end{center}
    So we have
    
    \begin{center}
        $ f(2k) \geq \frac{2k - \lambda_{2}}{2}$
    \end{center}
  
      If we take expectation over $\mathscr{G}_{n, 2 d}$ on both side we get 
      \begin{center}
           $ f(2k) \geq \frac{2k - E(\lambda_{2})}{2}$
      \end{center}
      and as $\mathrm{E|\lambda_{2}(G)|}
    \leq O(k^{1/2})$ we have 
 \begin{center}
    $f(2k) \geq \frac{2k - O(k^{1/2})}{2} \geq k - O(k^{1/2}) $ 
 \end{center}
    
  \end{proof}  
    
  
    \addcontentsline{toc}{chapter}{Bibliography}
\begin{thebibliography}{300}
\bibitem{spherical_psf} \label{t:21} 
Wolfgang Weil: A Course
on
Convex Geometry
 \bibitem{spherical_psf}
Issac Chavel: Isoperimetric Inequalities (Differetial Geometric and Analytic Perspective)
\bibitem{spherical_psf}
Keith Ball:An Elementry Introduction to Modern Convex Geometry
University of Karlsruhe
2002/2003
revised version 2004/2005
\bibitem{spherical_psf}
     BOJAN MOHAR: Isoperimetric numbers of Graph.
     
    
    \bibitem{spherical_psf} JOEL FRIEDMAN: On the Second Eigenvalue and Random Walks in Random d-regular Graphs.
     \bibitem{spherical_psf}
     
     ANDREI BRODER, and ELI SHAMIR: On the second eigenvalue of random regular graphs. In 28th annual Symposium on Foundations of Computer Science, pages 286-294, 1987. 
     
     \bibitem{spherical_psf}
      N.ALON and V.D. MILMAN,$\lambda_{1}$ isoperimetric inequalities for graphs, and super concentrator, J Combin Theory Ser.B38(1985),73-88  
  \end{thebibliography}  
    
       
    
 
 
 
 
 
 
     
       
       
       
       
       
       
       
       
       
       
       
       
       
       
       
       
       
       
       
       
       
       
       
       
       
       
       
       
       
       
       
       
       
       
       
       
       
       
       
       
       
       
       
       
       
       
       
       
       
       
       
       
       
       
       
       
       
       
       
       
       
       
       
       
       
       
       
       
       
       
       
     
\end{document}
